\chapter{ANÁLISE EMPÍRICA}\label{cap-analise-empirica}

Os testes foram executados num computador com processador Ryzen 5 3600X rodando a 4.0GHz e 16GB de RAM. Cada entrada foi executada 50 vezes, sendo tirada a média das 50 iterações. As porcentagens foram arredondadas para 1 casa decimal e podem não totalizar 100\%.
\renewcommand\cellgape{\Gape[2pt]}
\begin{table}[h]
\centering
\begin{tabular}{lccccccc}
\toprule
{} &   Leitura &  Distâncias &  Ordenação &       MST &  Agrupamento &   Escrita \\

Entrada 1 & \mc{0.000066 \\ (23.9\%)}
&   \mc{0.000007 \\ (2.5\%)}
&   \mc{0.000115 \\ (41.7\%)}
&   \mc{0.000009 \\ (3.3\%)}
&   \mc{0.000003 \\ (1.0\%)}
&   \mc{0.000076 \\ (27.5\%)}
\\
Entrada 2 & \mc{0.000107 \\ (14.2\%)}
&   \mc{0.000031 \\ (4.2\%)}
&   \mc{0.000517 \\ (69.3\%)}
&   \mc{0.000021 \\ (2.8\%)}
&   \mc{0.000006 \\ (0.8\%)}
&   \mc{0.000064 \\ (8.6\%)}
\\
Entrada 3 & \mc{0.000524 \\ (0.6\%)}
&   \mc{0.002971 \\ (3.5\%)}
&   \mc{0.079349 \\ (94.9\%)}
&   \mc{0.000287 \\ (0.3\%)}
&   \mc{0.000272 \\ (0.3\%)}
&   \mc{0.000217 \\ (0.3\%)}
\\
Entrada 4 & \mc{0.002632 \\ (0.4\%)}
&   \mc{0.022593 \\ (3.8\%)}
&   \mc{0.565981 \\ (95.3\%)}
&   \mc{0.000761 \\ (0.1\%)}
&   \mc{0.000926 \\ (0.2\%)}
&   \mc{0.000371 \\ (0.1\%)}
\\
Entrada 5 & \mc{0.009535 \\ (0.4\%)}
&   \mc{0.115482 \\ (4.5\%)}
&   \mc{2.453364 \\ (95.0\%)}
&   \mc{0.001777 \\ (0.1\%)}
&   \mc{0.002072 \\ (0.1\%)}
&   \mc{0.000649 \\ (0.0\%)}
\\


\bottomrule
\end{tabular}
\caption{Tabela com tempos de execução em segundos para cada entrada fornecida.}
\label{tab-tempo-execucao}
\end{table}


Podemos notar que, para as entradas 1 (50 pontos, $m = 2$) e 3 (1000 pontos, $m = 2$, 20x maior):
\vspace{-1em}
\begin{itemize}
\setlength\itemsep{0em}

\item O cálculo de distâncias é 424x mais demorado, o que condiz com o valor de $O(n^2)$ obtido, visto que uma entrada 20x maior terá 400x mais impacto no tempo de execução deste trecho;
\item Ordenação de distâncias cresce 690x, também condizendo com o valor de $O(N \log{N})$, uma vez que este cresce \~{}750x ao aumentar $n$ em 20x, tendo $N = \frac{n(n-1)}{2}$;
\item Da entrada 3 para 5 (5x mais pontos), o agrupamento tornou-se 7.6x mais lento, condizendo com $O(n \log{n})$.

\end{itemize}