\documentclass{sample}

% Enumera equações de acordo com a seção
% \numberwithin{equation}{section}

\begin{document}

In first-year calculus, we define intervals such as $(u, v)$ and $(u, \infty)$. Such an interval
is a neighborhood of $a$ if $a$ is in the interval. Students should realize that $\infty$ is only
a symbol, not a number. This is important since we soon introduce concepts such as $\lim_{x \to \infty} f(x)$.

When we introduce the derivative
\[ % Não usar $$
    \lim_{x \to a}\frac{f(x) - f(a)}{x - a}
\]
we assume that the function is defined and continuous in a neighborhood of $a$.

% Enésima raiz (argumentos opcionais ficam entre colchetes):
\[
    \sqrt[2832]{abc}
\]

% Exemplo de matriz
\[
    \begin{bmatrix}
        a + b + c & uv    & x - y & 27  \\
        a + b     & u + v & z     & 134
    \end{bmatrix}
\]

% Exemplo de matriz com linhas verticais e rótulo
\begin{equation}\label{E:matrizV}
    \begin{vmatrix}
        a + b + c & uv    & x - y & 27  \\
        a + b     & u + v & z     & 134
    \end{vmatrix}
\end{equation}

% Exemplo de matriz com parênteses e tag arbitrária
\begin{equation}\label{E:matrizP}\tag{tag arbitrária}
    \begin{pmatrix}
        a + b + c & uv    & x - y & 27  \\
        a + b     & u + v & z     & 134
    \end{pmatrix}
\end{equation}


Veja (\ref{E:matrizV}) na página \pageref{E:matrizV}.

Veja (\ref{E:matrizP}) na página \pageref{E:matrizP}.

% Alinhamento de fórmulas (usar apenas um & para definir pontos de alinhamento)
% & pode ser usado para dividir equações em múltiplas linhas,
% usando o mesmo environment (usar \notag)
% align*: sem numeração
\begin{align}
    r^2 &= s^2 + t^2,\\
    2u + 1 &= v + w^{\alpha},\\
    x &= \frac{y + z}{\sqrt{s + 2u}}.
\end{align}


% Casos:
\[
    f (x) =
    \begin{cases}
        -x^{2}, &\text{if $x < 0$;}\\
        a + x,  &\text{if $0 \leq x \leq 1$;}\\
        x^2,    &\text{otherwise.}
    \end{cases}
\]

\end{document}
