\documentclass[
	% --- opções da classe memoir ---
	12pt,				% Tamanho da fonte
	openright,			% Capítulos começam em pág ímpar (insere página vazia caso preciso)
	oneside,			% Para impressão em recto e verso. Oposto a oneside
	a4paper,			% Tamanho do papel.
	% -- opções do pacote babel --
	english,			% Idioma adicional para hifenização
% 	french,				% Idioma adicional para hifenização
% 	spanish,			% Idioma adicional para hifenização
	brazil,				% O último idioma é o principal do documento
	]{abntex2}


% ---
% Pacotes
% ---
\usepackage{lmodern}			% Usa a fonte Latin Modern
\usepackage[T1]{fontenc}		% Seleção de códigos de fonte.
\usepackage[utf8]{inputenc}		% Codificação do documento (conversão automática dos acentos)
\usepackage{indentfirst}		% Indenta o primeiro parágrafo de cada seção.
\usepackage{color}				% Controle das cores
\usepackage{graphicx}			% Inclusão de gráficos
\usepackage{microtype} 			% Para melhorias de justificação
\usepackage{float}
\usepackage[ampersand]{easylist}
\usepackage{enumitem}
\usepackage{titlesec}
\usepackage{caption}
\usepackage{subcaption}
\usepackage{multicol}
\usepackage{multirow}
\usepackage{lipsum}
\usepackage{amsfonts}
\usepackage{siunitx}
\usepackage{amsmath}
\usepackage{systeme}

\usepackage[brazilian,hyperpageref]{backref}
\usepackage[alf]{abntex2cite}

\setlist[enumerate]{itemsep=1pt}

%%%%%%%%%%%%%%%%%%%%%%%%%%%%%%%%%%%%%%%%%%%%%%%%%%
%% COLOR DEFINITIONS
%%%%%%%%%%%%%%%%%%%%%%%%%%%%%%%%%%%%%%%%%%%%%%%%%%
% \usepackage[svgnames]{xcolor}
% %%%%%%%%%%%%%%%%%%%%%%%%%%%%%%%%%%%%%%%%%%%%%%%%%%
% \definecolor{MyColor1}{rgb}{0.2,0.4,0.6} %mix personal color
% \newcommand{\textb}{\color{Black} \usefont{OT1}{lmss}{m}{n}}
% \newcommand{\blue}{\color{MyColor1} \usefont{OT1}{lmss}{m}{n}}
% \newcommand{\blueb}{\color{MyColor1} \usefont{OT1}{lmss}{b}{n}}
% \newcommand{\red}{\color{LightCoral} \usefont{OT1}{lmss}{m}{n}}

\renewcommand\thesection{\arabic{section}.} %define sections numbering
\renewcommand\thesubsection{\thesection\arabic{subsection}} %subsec.num.

\newcommand{\mysection}{
\titleformat{\section} [runin] {\usefont{OT1}{lmss}{b}{n}\color{MyColor1}} 
{\thesection} {3pt} {} } 

\renewcommand{\theequation}{\thesection\arabic{equation}}

% ---
% Informações de dados para CAPA e FOLHA DE ROSTO
% ---
\titulo{\textmd{Estatística Básica}\\
Lista de Exercícios Extra}
\autor{Atílio Antônio Dadalto}
\local{Vitória}
\data{2019}
\instituicao{%
  Universidade Federal do Espírito Santo
  \par Departamento de Informática}
\tipotrabalho{Relatório}
\preambulo{Lista de exercícios apresentada como requisito parcial para aprovação na disciplina de Estatística Básica, pela Universidade Federal do Espírito Santo.}
% \let\cleardoublepage\clearpage

\newcommand{\versao}{2.0}
\newcommand{\subtitulo}{Anteprojeto}


%%% Configurações finais de aparência. %%%
% Altera o aspecto da cor azul.
% \definecolor{blue}{RGB}{41,5,195}

% Informações do PDF.
\makeatletter
\hypersetup{
	pdftitle={\@title}, 
	pdfauthor={\@author},
	pdfsubject={\imprimirpreambulo},
	pdfcreator={LaTeX with abnTeX2},
	pdfkeywords={abnt}{latex}{abntex}{abntex2}{trabalho acadêmico}, 
	colorlinks=true,				% Colore os links (ao invés de usar caixas).
	linkcolor=black,					% Cor dos links.
	citecolor=blue,					% Cor dos links na bibliografia.
	filecolor=magenta,				% Cor dos links de arquivo.
	urlcolor=blue,					% Cor das URLs.
	bookmarksdepth=4
}
\makeatother

% Espaçamentos entre linhas e parágrafos.
\setlength{\parindent}{1.3cm}
\setlength{\parskip}{0.2cm}


\begin{document}
% Capa do trabalho.
\imprimircapa
% \frenchspacing
\textual

\section{Solução da Questão 1}
$Y$ é uma variável que indica o tempo entre ocorrências. Sabemos que:
\begin{align*}
    F(y) &= P(Y < y)\\
    &= 1 - P(Y > y)
\end{align*}
$P(Y > y)$ é a probabilidade do intervalo entre dois eventos ser maior que $y$, isto é, a probabilidade de não ocorrer nenhum evento em um intervalo $y$.

Seja $t = y$. Podemos representar a mesma situação com uma distribuição Poisson com parâmetro $\lambda y$ (apenas variando o intervalo de interesse) em que $X = 0$ (não ocorrerá nenhum evento). Isto é, $P(Y > y) = P(X = 0)$. Então:
\begin{align*}
    F(y) &= 1 - P(Y > y)\\
    &= 1 - P(X = 0)\\
    &= 1 - \frac{e^{-\lambda y}(\lambda y)^0}{0!}\\
    &= 1 - e^{-\lambda y}
\end{align*}
Então $P(Y < y) = 1 - P(X = 0)= 1 - e^{-\lambda y}$. Também sabemos que:
    $$Y \text{ segue uma distribuição exponencial}$$
    $$\iff$$
    $$F(y) = P(Y < y) = 1 - e^{-\lambda y}$$
Portanto, $Y$ segue uma distribuição exponencial de parâmetro $\lambda$.

\section{Solução da Questão 2}
Um equipamento será considerado fora de controle caso o desempenho se afaste da média em 2 unidades do desvio padrão. Isto é, a probabilidade de que o equipamento seja considerado fora de controle é igual a $P(X > \mu + 2\sigma) + P(X < \mu - 2\sigma)$. Levando para a forma normal reduzida:
\begin{align*}
    P(X > \mu + 2\sigma) &= P\left(\frac{X - \mu}{\sigma} > \frac{\mu + 2\sigma - \mu}{\sigma}\right)\\
    &= P(Z > 2) = 1 - P(Z < 2) = 1 - 0.9722 = 0.0228.\\[1em]
    P(X < \mu - 2\sigma) &= P\left(\frac{X - \mu}{\sigma} < \frac{\mu - 2\sigma - \mu}{\sigma}\right)\\
    &= P(Z < -2) = 0.0228.
\end{align*}
Então a probabilidade de que o equipamento seja considerado fora de controle é igual a 0.0228 + 0.0228 = 0.0456.

\begin{enumerate}[label=\alph*)]
    \item O equipamento será desligado caso seja avaliado e esteja fora de controle. As avaliações são independentes e queremos saber a probabilidade de que o primeiro sucesso --- estar fora de controle --- ocorra na 1ª tentativa. Isso sugere o uso de um modelo geométrico com $p = 0.0456$ (calculado acima):
    $$P(X = k) = p(1-p)^{k-1} \to P(X = 1) = 0.0456(0.9544)^0 = 0.0456.$$
    
    \item Temos a mesma situação do item a), porém com X = 10, visto que desejamos saber a probabilidade do primeiro sucesso, isto é, primeira manutenção decorrente do equipamento estar fora de controle, ser no décimo dia:
    $$P(X = k) = p(1-p)^{k-1} \to P(X = 10) = 0.0456(0.9544)^9 \approx 0.0299.$$
    
    \item Como mencionado acima, utilizamos uma variável seguindo o modelo geométrico para calcular a probabilidade do primeiro sucesso acontecer no $k$-ésimo dia, ou seja, ocorrerem $k - 1$ fracassos antes do primeiro sucesso (manutenção), com $p = 0.0456$.
\end{enumerate}
\end{document}