\documentclass[
	% --- opções da classe memoir ---
	12pt,				% Tamanho da fonte
	openright,			% Capítulos começam em pág ímpar (insere página vazia caso preciso)
	twoside,			% Para impressão em recto e verso. Oposto a oneside
	a4paper,			% Tamanho do papel.
	% -- opções do pacote babel --
	english,			% Idioma adonal para hifenização
	french,				% Idioma adicional para hifenização
	spanish,			% Idioma adicional para hifenização
	brazil,				% O último idioma é o principal do documento
	]{abntex2}


% ---
% Pacotes
% ---
\usepackage{lmodern}			% Usa a fonte Latin Modern
\usepackage[T1]{fontenc}		% Seleção de códigos de fonte.
\usepackage[utf8]{inputenc}		% Codificação do documento (conversão automática dos acentos)
\usepackage{indentfirst}		% Indenta o primeiro parágrafo de cada seção.
\usepackage{color}				% Controle das cores
\usepackage{graphicx}			% Inclusão de gráficos
\usepackage{microtype} 			% Para melhorias de justificação
\usepackage{float}
\usepackage{mathabx}
\usepackage[ampersand]{easylist}
\usepackage{enumitem}

\usepackage{multicol}
\usepackage{multirow}

\usepackage{lipsum}

\usepackage[brazilian,hyperpageref]{backref}
\usepackage[alf]{abntex2cite}

\setlist[enumerate]{itemsep=1pt}

%%%%%%%%%%%%%%%%%%%%%%%%%%%%%%%%%%%%%%%%%%%%%%%%%%
%% COLOR DEFINITIONS
%%%%%%%%%%%%%%%%%%%%%%%%%%%%%%%%%%%%%%%%%%%%%%%%%%
\usepackage[svgnames]{xcolor}
%%%%%%%%%%%%%%%%%%%%%%%%%%%%%%%%%%%%%%%%%%%%%%%%%%
\definecolor{MyColor1}{rgb}{0.2,0.4,0.6} %mix personal color
\newcommand{\textb}{\color{Black} \usefont{OT1}{lmss}{m}{n}}
\newcommand{\blue}{\color{MyColor1} \usefont{OT1}{lmss}{m}{n}}
\newcommand{\blueb}{\color{MyColor1} \usefont{OT1}{lmss}{b}{n}}
\newcommand{\red}{\color{LightCoral} \usefont{OT1}{lmss}{m}{n}}


\usepackage{titlesec}


\renewcommand\thesection{\arabic{section}.} %define sections numbering
\renewcommand\thesubsection{\thesection\arabic{subsection}} %subsec.num.

\newcommand{\mysection}{
\titleformat{\section} [runin] {\usefont{OT1}{lmss}{b}{n}\color{MyColor1}} 
{\thesection} {3pt} {} } 


\usepackage{caption}
\usepackage{subcaption}

\renewcommand{\theequation}{\thesection\arabic{equation}}

\title{\blue Estatística Básica \\
\blueb Lista de Exercícios 1}
\author{Atílio Antônio Dadalto\\Ezequiel Schneider Reinholtz\\Luana Gabriele de Sousa Costa}

\date{2019}
%%%%%%%%%%%%%%%%%%%%%%%%%%%%%%%%%%%%%%%%%%%%%%%%%%



\begin{document}
% Retira espaço extra obsoleto entre as frases.
\frenchspacing

\textual


\maketitle
\textual
\newpage
% \imprimirfolhaderosto

\section{Solução da Questão 1}
\subsection{Parte a)}
Temos as seguintes variáveis:
\begin{enumerate}[label=\Alph*)]
    \item dieta - qualitativa nominal;
    \item idade - quantitativa discreta;
    \item peso inicial - quantitativa contínua;
    \item peso após um mês - quantitativa contínua;
    \item IMC inicial - quantitativa contínua;
    \item IMC após um mês - quantitativa contínua.
\end{enumerate}

\subsection{Parte b)}
\begin{table}[ht]
    \centering
    \renewcommand\arraystretch{1.5}
    \begin{tabular}{c c c c c c c}
        \hline & PD0 A & PD30 A & PD0 B & PD30 B & PD0 C & PD30 C \\
        \hline Média & 74,3 & 71 & 79 & 75,5 & 83,8 & 83,3 \\
        Mediana & 72 & 69,5 & 78 & 73,5 & 86 & 85,5 \\
        Desvio Padrão & 7,98 & 9,29 & 12,68 & 12,15 & 13,58 & 13,17 \\
        \hline
    \end{tabular}
    \caption{Medidas resumo de pesos das dietas}
    \label{medidas-pesos}
\end{table}

Analisando a Tabela~\ref{medidas-pesos}, temos que indivíduos emagreceram sob a dieta A, visto que a média e a mediana diminuíram. Os pesos também estão levemente mais distanciados da média. A dieta B apresentou a maior queda no peso dos participantes, a notar pela média e mediana. O grupo de controle C manteve o peso estável, com leve flutuação. 

\subsection{Parte c)}
Para a idade, poderíamos usar o boxplot, mas o pequeno intervalo de idades -- de 18 a 21 anos apenas -- em uma amostra pequena inviabiliza o uso desta visualização. Neste caso, podemos utilizar um histograma, dado que desejamos visualizar a distribuição de uma única variável quantitativa:
\begin{figure}[H]
	\begin{center}
	    \includegraphics[scale=0.75]{figuras/q1c.pdf}
	\end{center}
\caption{Histograma demonstrando a densidade de idades}
\end{figure}
Sendo assim, fica claro que as idades se encontram no intervalo $[18, 21]$.

\subsection{Parte d)}
\begin{figure}[!htbp]
	\begin{center}
	    \includegraphics[scale=0.75]{figuras/q1d.pdf}
	\end{center}
\caption{Gráfico de dispersão entre IMCD0 e IMCD30 para as três dietas}
\end{figure}

Por meio da visualização do gráfico de dispersão, fica muito mais clara a comparação entre as três dietas. As dietas A e B ambas apresentaram queda de IMC dos participantes em relação à C, com desempenho parecido entre as duas neste conjunto de dados e a A tendo melhor expectativa para perda de IMC.

\section{Solução da Questão 2}
\subsection{Parte a)}
A população, uma vez que o gestor usou todo o conjunto de funcionários para fazer o teste de eficiência, em vez de um conjunto menor que a população.

\subsection{Parte b)}
Cada medida resumo reflete o resultado dos grupos (A e B) em dois momentos diferentes (T1 e T2):
\begin{enumerate}[label=\Alph*)]
    \item mínimo - a menor nota 
    \item 1º quartil - a nota que está acima de 25\% das notas
    \item média - nota que ``equilibra'' as frequências das notas
    \item mediana - a nota que está acima de 50\% das notas
    \item 3º quartil - a nota que está acima de 75\% das notas
    \item máximo - a maior nota
    \item desvio padrão - o quanto as notas se desviaram da média
\end{enumerate}

\subsection{Parte c)}
O grupo A demonstrou significativa melhoria nas medidas resumo em T2, o que atesta a eficiência do curso.

\subsection{Parte d)}
No primeiro momento, não há muita diferença entre os dois grupos A e B. No entanto, após aplicado o curso, o grupo A demonstra um ganho de nota enorme, com o intervalo interquartil do grupo A em T2 muito acima do intervalo interquartil do grupo B no mesmo momento.

\subsection{Parte e)}
Sim, pois a diferença entre o grupo de controle e o de tratamento é expressiva --- todas as medidas resumo apresentaram melhoria, e a mediana das notas quase dobrou.

\section{Solução da Questão 3}
\subsection{Parte a)}
Em uma situação em que analisa-se a renda de uma população, visto que a mediana é mais robusta contra os poucos pontos de renda muito alta que enviesariam a média e, portanto, é mais apropriada para representar a renda da população.
\subsection{Parte b)}
\begin{figure}[!htbp]
	\begin{center}
	    \includegraphics[scale=0.85]{figuras/q3b.pdf}
	\end{center}
\caption{Exemplo de histograma de dados com média = mediana = 5.50.}
\end{figure}
Distribuições simétricas sempre possuem média igual à mediana.

\subsection{Parte c)}
\begin{figure}[H]
	\begin{center}
	    \includegraphics[scale=0.65]{figuras/q3c.pdf}
	\end{center}
\caption{Exemplo de histograma de dados com média = 5.36 e mediana = 8.50.}
\end{figure}
Uma situação em que isso ocorre é com a idade gestacional no parto, uma vez que a distribuição é assimétrica para a esquerda.

\subsection{Parte d)}
\begin{figure}[!htbp]
	\begin{center}
	    \includegraphics[scale=0.7]{figuras/variavelX.pdf}
	\end{center}
\caption{Variável X com variância = 5.83 e média = 5.00}
\end{figure}

\begin{figure}[!htbp]
	\begin{center}
	    \includegraphics[scale=0.75]{figuras/variavelY.pdf}
	\end{center}
\caption{Variável Y com variância = 7.17 e média = 5.00}
\end{figure}

\begin{figure}[!htbp]
	\begin{center}
	    \includegraphics[scale=0.75]{figuras/variavelZ.pdf}
	\end{center}
\caption{Variável Z com variância = 10.88 e média = 5.00}
\end{figure}



\end{document}