\documentclass[
	% --- opções da classe memoir ---
	12pt,				% Tamanho da fonte
	openright,			% Capítulos começam em pág ímpar (insere página vazia caso preciso)
	oneside,			% Para impressão em recto e verso. Oposto a oneside
	a4paper,			% Tamanho do papel.
	% -- opções do pacote babel --
	english,			% Idioma adicional para hifenização
% 	french,				% Idioma adicional para hifenização
% 	spanish,			% Idioma adicional para hifenização
	brazil,				% O último idioma é o principal do documento
	]{abntex2}


\usepackage{lmodern}			% Usa a fonte Latin Modern
\usepackage[T1]{fontenc}		% Seleção de códigos de fonte.
\usepackage[utf8]{inputenc}		% Codificação do documento (conversão automática dos acentos)
\usepackage{indentfirst}		% Indenta o primeiro parágrafo de cada seção.
\usepackage{color}				% Controle das cores
\usepackage{graphicx}			% Inclusão de gráficos
\usepackage{microtype} 			% Para melhorias de justificação
\usepackage{float}
\usepackage[ampersand]{easylist}
\usepackage{enumitem}
\usepackage{titlesec}
\usepackage{caption}
\usepackage{subcaption}
\usepackage{multicol}
\usepackage{multirow}
\usepackage{lipsum}
\usepackage{amsfonts}
\usepackage{siunitx}
\usepackage{amsmath}
\usepackage{systeme}
\usepackage{mathtools}

\usepackage[brazilian,hyperpageref]{backref}
\usepackage[alf]{abntex2cite}

\setlist[enumerate]{itemsep=1pt}


\renewcommand\thesection{\arabic{section}.} %define sections numbering
\renewcommand\thesubsection{\thesection\arabic{subsection}} %subsec.num.

\newcommand{\mysection}{
\titleformat{\section} [runin] {\usefont{OT1}{lmss}{b}{n}\color{MyColor1}} 
{\thesection} {3pt} {} } 

\renewcommand{\theequation}{\thesection\arabic{equation}}

% ---
% Informações de dados para CAPA e FOLHA DE ROSTO
% ---
\titulo{\textmd{Estatística Básica}\\
Lista de Exercícios 5}
\autor{Atílio Antônio Dadalto \\ Tiago da Cruz Santos}
\local{Vitória}
\data{2019}
\instituicao{%
  Universidade Federal do Espírito Santo
  \par Departamento de Informática}
\tipotrabalho{Relatório}
\preambulo{Relatório apresentado como requisito parcial para aprovação na disciplina de Estrutura de Dados I, pela Universidade Federal do Espírito Santo.}
% \let\cleardoublepage\clearpage

\newcommand{\versao}{2.0}
\newcommand{\subtitulo}{Anteprojeto}


%%% Configurações finais de aparência. %%%
% Altera o aspecto da cor azul.
% \definecolor{blue}{RGB}{41,5,195}

% Informações do PDF.
\makeatletter
\hypersetup{
	pdftitle={\@title}, 
	pdfauthor={\@author},
	pdfsubject={\imprimirpreambulo},
	pdfcreator={LaTeX with abnTeX2},
	pdfkeywords={abnt}{latex}{abntex}{abntex2}{trabalho acadêmico}, 
	colorlinks=true,				% Colore os links (ao invés de usar caixas).
	linkcolor=black,					% Cor dos links.
	citecolor=blue,					% Cor dos links na bibliografia.
	filecolor=magenta,				% Cor dos links de arquivo.
	urlcolor=blue,					% Cor das URLs.
	bookmarksdepth=4
}
\makeatother

% Espaçamentos entre linhas e parágrafos.
\setlength{\parindent}{1.3cm}
\setlength{\parskip}{0.2cm}

\begin{document}
\imprimircapa
% \frenchspacing
\textual

\section{Solução da Questão 1}
\begin{enumerate}[label=\alph*)]
    \item Queremos calcular $IC(\mu, 0.90)$ e $IC(\mu, 0.90)$:
    
    \begin{align*}
        IC(\mu, 0.90) &= \left(18.5 - 1.65 \cdot \frac{6}{\sqrt{50}};18.5 + 1.65 \cdot \frac{6}{\sqrt{50}}\right)\\
        &= (17.10, 19.90)\\
        IC(\mu, 0.99) &= \left(18.5 - 2.58 \cdot \frac{6}{\sqrt{50}};18.5 + 2.58 \cdot \frac{6}{\sqrt{50}}\right)\\
        &= (16.31, 20.69)
    \end{align*}
    É notável que a margem de erro e o grau de confiança são medidas que crescem juntas,  portanto faz sentido o erro aumentar conforme aumenta-se o grau de confiança de 90\% para 99\%.
    
    \item Queremos calcular $IC(\mu, 0.90)$ para $n = 25$ e $n = 100$:
    
    Para $n = 25$:
    \begin{align*}
        IC(\mu, 0.90) &= \left(18.5 - 1.65 \cdot \frac{6}{\sqrt{25}};18.5 + 1.65 \cdot \frac{6}{\sqrt{25}}\right)\\
        &= (20.48, 16.52)
    \end{align*}
    
    Para $n = 100$:
    \begin{align*}
        IC(\mu, 0.90) &= \left(18.5 - 1.65 \cdot \frac{6}{\sqrt{100}};18.5 + 1.65 \cdot \frac{6}{\sqrt{100}}\right)\\
        &= (17.51, 19.49)
    \end{align*}
    
    Temos que, pelo TLC, conseguimos uma precisão maior com amostras maiores, portanto faz sentido o intervalo com $n = 100$ ser mais preciso do que o intervalo com $n = 25$.

\end{enumerate}

\section{Solução da Questão 2}

Dadas as informações da questão, as seguintes informações estão presentes:

\begin{align*}
    \bar{x} &= 50 &H_{0}: \mu = 60\\
    \sigma &= 20 &H_{1}: \mu \neq 60\\
    \mu &= 60\\
    \alpha &= 0.05\\
    n &= 9
\end{align*}

Sabemos que $z_{1 - \frac{\alpha}{2}} = 1.96$, pela tabela da normal. Calculando a estatística do teste:
\begin{align*}
Z_{obs} &= \frac{\bar{x} - \mu_{0}}{\frac{\sigma}{\sqrt{n}}}\\
&= \frac{50 - 60}{\frac{20}{\sqrt{9}}}\\
&= -1.5
\end{align*}

Como o teste é bilateral, a região crítica é $Z < -1.96$ e $Z > 1.96$. Sabemos que $-1.96 < Z < 1.96$, portanto não temos evidências para rejeitar $H_{0}$.

\section{Solução da Questão 3}

\begin{enumerate}[label=\alph*)]
    \item Queremos calcular $IC(\mu, 0.95)$:
    
    \begin{align*}
        IC(\mu, 0.95) &= \left(31.5 - 1.96 \cdot \frac{3}{\sqrt{25}};31.5 + 1.96 \cdot \frac{3}{\sqrt{25}}\right)\\
        &= (30.325, 32.675)
    \end{align*}
    
    \item Definindo nossas hipóteses para $\mu \coloneqq$ quantidade média de nicotina nos cigarros produzidos pelo fabricante B: 
    \[
        \begin{cases}
            H_0: \mu \leq 30\\
            H_1: \mu > 30
        \end{cases}
    \]
    
    Fixamos o nível de significância em 5\%.
    
    Agora, a estatística do teste:
    \begin{align*}
        Z_{obs} &= \frac{\bar{x} - \mu_0}{\frac{\sigma}{\sqrt{n}}} \sim N(0, 1)\\
        &= \frac{31.5 - 30}{\frac{3}{\sqrt{25}}}\\
        &= 0.5 \cdot 5\\
        &= 2.5
    \end{align*}
    
    Definindo região crítica: como o teste é unilateral à direita com nível de significância 5\%, a região crítica é todo $n \in \mathbb{R}$ tal que $n > 1.96$.
    
    $Z_{obs} = 2.5$ e $2.5 > 1.96$, portanto rejeitamos $H_0$. Logo, há indícios de que o fabricante A tem razão em sua afirmação de que os cigarros do fabricante B possuem mais do que 30mg de nicotina.
\end{enumerate}


\section{Solução da Questão 4}

\begin{enumerate}[label=\alph*)]
    \item A estimativa pontual para $p$ é de $\frac{13}{80} = 16.25\%$
    \item 
    
    Ao usar a confiança como $90\%$, sabemos que $\alpha = 0.1$. Então $z_{1 - \frac{\alpha}{2}} = 1.645$ pela tabela da normal:
    
    \begin{align*}
        IC(p, 0.90) &= \left(0.1625 - 1.645\sqrt{\frac{0.1625(1 - 0.1625)}{80}}; 0.1625 + 1.645\sqrt{\frac{0.1625(1 - 0.1625)}{80}}\right)\\
        &= \left(0.09465;0.23034\right)
    \end{align*}
    \item Sabemos que, usando $\alpha = 10\%$, $z_{1 - \frac{\alpha}{2}} = 1.645$. Calculando a estatística do teste, temos que:
    \begin{align*}
        H_{0}: p &= 0.09\\
        H_{1}: p &\neq 0.09
    \end{align*}
    \begin{align*}
        Z_{obs} = \frac{0.1625 - 0.09}{\sqrt{\frac{0.09(1 - 0.09)}{80}}} = 2.266
    \end{align*}
    
    Como $1.645 < 2.266$, existem evidências para rejeitar $H_{0}$, ou seja, a proporção de abortos espontâneos em grávidas que consomem quantidades elevadas de cafeína é diferente da proporção de abortos espontâneos em grávidas normais.
    
\end{enumerate}


\section{Solução da Questão 5}

\begin{enumerate}[label=\alph*)]
    \item
    \[
        \begin{cases}
            H_0: \mu \geq 90\\
            H_1: \mu < 90
        \end{cases}
    \]
    \item Há indícios de que a afirmação da SSA é falsa, uma vez que o valor $p$ menor que o nível de significância usual (0.05) nos faz questionar a validade da hipótese nula.
    \item Sim, pois, ainda que existam indícios contra a hipótese nula, estes não são tão fortes, e, com um preço menor, torna-se mais desejável utilizar este medicamento.
\end{enumerate}


\section{Solução da Questão 6}

Para refutar a afirmação, o fabricante pode fazer um teste de hipóteses onde $H_{0}$ representa a proporção de produtos defeituosos ser menor ou igual a $20\%$:

\begin{align*}
    n &= 50 &H_{0}: p \leq 0.2\\
    \hat{p} &= 0.27 &H_{1}: p > 0.2\\
    \alpha &= 0.05
\end{align*}

Pela tabela da normal, sabemos que $z_{\alpha} = 1.645$. Calculando a estatística do teste, temos que:

\begin{align*}
    \frac{\hat{p} - p}{\sqrt{\frac{p(1 - p)}{n}}} &= \frac{0.27 - 0.2}{\sqrt{\frac{0.2(1 - 0.2)}{50}}}\\
    &= 1.2374
\end{align*}

Como o teste é unilateral à direita, a região crítica é  $Z > 1.645$. Visto que $Z < 1.645$, não existem evidências para rejeitar $H_{0}$.

\section{Solução da Questão 7}

\begin{enumerate}[label=\alph*)]
    \item Podemos notar visualmente uma correlação positiva entre o eixo x e o y:
    \begin{figure}[H]
        \centering
        \includegraphics[scale = 0.75]{figuras/dispersao7.pdf}
        \caption{Gráfico de dispersão}
    \end{figure}
    \item Sabemos que o coeficiente linear $r$ de correlação de Pearson é dado por
    \begin{align*}
        r &= \frac{n(\sum xy) - (\sum x)(\sum y)}{\sqrt{n\sum x^2 - (\sum x)^2][n\sum y^2 - (\sum y)^2]}}\\
        &\approx 0.98
    \end{align*}
    Portanto, existe uma correlação muito forte entre a média final e o ganho por horas trabalhadas.
    \item Temos que $\alpha \approx 1.72$ e $\beta \approx 2.83$. Portanto, a reta é aproximadamente $y = 1.72 + 2.83x + \epsilon$.
    \begin{figure}[H]
        \centering
        \includegraphics[scale = 0.75]{figuras/dispersao7fit.pdf}
        \caption{Gráfico com reta ajustada}
    \end{figure}
    \item $1.72 + 2.83\cdot 7 \approx \text{R}\$21.5$.
\end{enumerate}


\section{Solução da Questão 8}

\begin{enumerate}[label=\alph*)]
    \item Temos que $\alpha \approx 7.15$ e $\beta \approx -0.27$. Portanto, a reta é aproximadamente $y = 7.15 - 0.27x + \epsilon$.
    \item O modelo parece fazer sentido:
    \begin{figure}[H]
        \centering
        \includegraphics[scale = 0.75]{figuras/MMQ8.pdf}
        \caption{Gráfico com reta ajustada}
    \end{figure}
    \item $\beta$, é a inclinação da reta. Indica que a tendência do preço é diminuir conforme tomamos casas com idades maiores.
    \item É o ponto em que a reta intercepta o eixo $y$. Para uma casa recém-construída, com $x = 0$, temos que o valor assume $y = 7.15$.
\end{enumerate}


\end{document}