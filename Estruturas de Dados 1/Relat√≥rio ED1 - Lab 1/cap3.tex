\chapter{Análise e resultados}\label{cap-analise-resultado}

\section{Desempenho}\label{sec-desempenho}
Os gráficos foram produzidos por meio do Google Planilhas, aproveitando o fato de que os dados foram inicialmente dispostos no serviço. Os eixos possuem escala logarítmica:

\begin{center}\label{graf_ins_rand}
    \includegraphics[scale=0.75]{Relatório/figuras/insercao_randomico.pdf}
\end{center}

\begin{center}\label{graf_busca_rand}
    \includegraphics[scale=0.75]{Relatório/figuras/busca_randomico.pdf}
\end{center}


\begin{center}\label{graf_ins_seq}
    \includegraphics[scale=0.75]{Relatório/figuras/insercao_sequencial.pdf}
\end{center}

\begin{center}\label{graf_busca_seq}
    \includegraphics[scale=0.75]{Relatório/figuras/busca_sequencial.pdf}
    \small{Fonte: elaborado pelo autor.}
\end{center}

\section{Análise}
Com os dados, é possível analisar o desempenho das estruturas nas variadas circunstâncias aqui propostas.
Faz-se necessário comentar os resultados de entradas randômicas e sequenciais separadamente, devido à enorme discrepância no tempo de execução entre cada categoria até mesmo entre uma única estrutura de dados, como já foi visto nas figuras da Seção~\ref{sec-desempenho} para a árvore binária em inserção sequencial em comparação com a inserção randômica.

\section{Entradas randômicas}

\subsection{Lista encadeada com entradas randômicas}\label{sub_lista_entrada_r}
A lista encadeada de longe foi a que melhor performou para inserções; isto se deve à implementação muito simples da função de inserir, que é $O(1)$.
Em buscas, no entanto, ela é extremamente ineficiente em comparação com as árvores, já que sua busca é muito mais custosa e complexa, sendo até 1000 vezes mais lenta para buscas com 1,000,000 elementos se comparado às árvores binárias.
Portanto, para projetos de muitas inserções e poucas buscas, a lista encadeada é promissora, até mesmo por ter fácil implementação e manutenção.

\subsection{Árvore binária com entradas randômicas}
A árvore binária com entradas randômicas é pouco prejudicada pela falta de balanceamento em decorrência da distribuição aleatória dos dados de entrada. Seu desempenho é semelhante à da árvore balanceada neste caso, sendo ligeiramente pior para buscas.

\subsection{Árvore AVL com entradas randômicas}
A árvore balanceada também não enfrenta muitos problemas com entradas randômicas, já que o balanceamento é esporádico se comparado ao contexto de entradas sequenciais.
Os desempenhos das árvores são semelhantes entre si e cerca de 10 vezes inferiores ao da lista encadeada para inserções com entradas de 1,000,000 elementos.

\section{Entradas sequenciais}

\subsection{Lista encadeada com Entradas sequenciais}
Não há diferença notável entre os testes com entradas randômicas e entradas sequenciais para listas encadeadas.
Ver Seção~\ref{sub_lista_entrada_r}.

\subsection{Árvore binária com Entradas sequenciais}
É com entradas sequenciais que as árvores binárias têm maior dificuldade, em especial se não são balanceadas.
A árvore binária, recebendo dados sempre em ordem crescente, torna-se uma árvore degenerada, assemelhando-se à uma lista e ainda demonstrando um desempenho inferior a esta para entradas maiores que 10,000.

Apesar de ter um desempenho espetacular para buscas em entradas randômicas, no caso sequencial a árvore não conseguiu inserir mais do que cerca de 100,000 elementos, pois, com sua implementação recursiva, houve \textit{stack overflow} decorrente principalmente de seu total desbalanceamento.
Portanto, dados sobre 1,000,000 inserções estão indisponíveis, e seu desempenho acima de $n = 100$ já encontra-se bastante deteriorado.

\subsection{Árvore AVL com Entradas sequenciais}
Mesmo tendo de balancear-se regularmente, a árvore AVL obteve bons resultados para entradas sequenciais. Nas inserções, perde para a lista encadeada, mas ainda garante a melhor busca de entradas sequenciais dentre as três estruturas, com uma margem de 1000 vezes mais velocidade que a lista.