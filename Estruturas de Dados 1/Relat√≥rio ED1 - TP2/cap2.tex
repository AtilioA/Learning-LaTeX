\chapter{Implementação e funcionamento}\label{cap:implementacao-funcionamento}

Para a implementação do indexador, foram utilizadas diversas estruturas vistas no curso, desde listas encadeadas até tabelas de dispersão. As implementações foram baseadas nas notas da disciplina disponíveis no AVA\cite{vm-ava}; nas seções que seguem é discorrido acerca da composição dessas estruturas. O projeto conta com um total de 7 bibliotecas, sendo 5 focadas nas estruturas e 2 no indexador propriamente dito.

\section{Palavra}
Primeiramente, a estrutura que possivelmente seja a mais importante do projeto, a Palavra, é definida a seguir:
\begin{lstlisting}
/* Estrutura de dados para guardar nomes de arquivos e ocorrências de palavras nestes arquivos */
typedef struct Arq
{
    char *nomeArquivo;
    ListaOcorre *ocorrencias;
    struct Arq *prox;
} Arq;

typedef struct ListaArq
{
    Arq *primeiro;
    Arq *ultimo;
} ListaArq;

/* Estrutura de dados que abstrai uma palavra do texto */
typedef struct Palavra
{
    char *string;
    ListaArq *arquivos;
} Palavra;
\end{lstlisting}

Com ela, podemos armazenar um vetor de caracteres, que será uma palavra do texto, o nome do arquivo em que a palavra se encontra e as posições das ocorrências neste arquivo. Dessa forma, podemos ter conhecimento de ocorrências de uma palavra em um arquivo, além de suas posições, tudo em uma mesma estrutura. O TAD Palavra foi utilizado para todas as cinco estruturas de dados de indexação principais e conta com uma lista encadeada de nomes de arquivos, em que cada célula da lista carrega o nome de arquivo e uma lista encadeada de ocorrências da palavra no arquivo.

\section{Listas Encadeadas}
Primeiramente, uma das nossas estruturas mais simples, a lista encadeada, é definida a seguir:
\begin{lstlisting}
// Lista encadeada que abstrai um conjunto de palavras
    typedef struct Celula
    {
        Palavra *palavra;
        struct celula *prox;
    } Celula;

    typedef struct Lista
    {
        Celula *primeiro, *ultimo;
    } Lista
\end{lstlisting}

O ponto forte da lista encadeada de indexação seriam suas inserções $O(1)$. No entanto, para não inserir elementos (palavras) repetidos, devemos percorrer a lista em busca de um elemento igual ao de entrada; isso faz com que a inserção seja $O(n)$, algo que afetará o desempenho de forma visível no Capítulo~\ref{cap-analise-resultado}. Além dessa implementação, temos algumas outras variações, criadas para suportar a busca aleatória de palavras, além da utilizada para nomes de arquivos e para a lista de ocorrências na estrutura Palavra:

\begin{lstlisting}
/* Lista encadeada para armazenar ocorrências */
typedef struct ocorre
{
    int ocorreu;
    struct ocorre *prox;
} CelulaOcorre;

typedef struct ocorrencias
{
    CelulaOcorre *primeiro;
    CelulaOcorre *ultimo;
    int qtd;
} ListaOcorre;
\end{lstlisting}

\subsection{Lista encadeada de palavras aleatórias}

Para realizar as buscas aleatórias nas estruturas, a solução utilizada foi criar uma lista encadeada separada de strings:
\begin{lstlisting}
// Lista encadeada para pesquisar palavras aleatoriamente
typedef struct CelulaRandPal
{
    char *string;
    struct pal_rand *prox;
} CelulaRandPal;

typedef struct ListaRandPal
{
    CelulaRandPal *primeiro;
    CelulaRandPal *ultimo;
    int qtd;
} ListaRandPal;
\end{lstlisting}

Essa lista é preenchida durante a leitura, e, por ter inserção é $O(1)$, seu tempo não chega a interferir na inserção das estruturas estudadas. Após ser preenchida, uma palavra aleatória é buscada na lista e passada para um vetor de strings com tamanho igual ao número de buscas.

\begin{lstlisting}
printf("Preenchendo vetor de palavras aleatorias para busca...\n");
for (i = 0; i < n; i++)
{
    // Carregando o vetor com palavras aleatórias da lista
    strcpy(vetBusca[i], busca_RandPal(rand() % (palavrasAleatorias->qtd), palavrasAleatorias));
}
printf("Vetor de palavras aleatorias preenchido.\n");
\end{lstlisting}

Realizar a operação de passar para o vetor de buscas pode ser custosa, mas como não é contabilizada, ela não interfere nos tempos dos testes. Após isso, para cada elemento do vetor de busca, buscamos esse elemento na estrutura desejada, imprimindo o resultado.
\section{Árvores Binárias Não Balanceadas}\label{tadArvBin}
\begin{lstlisting}
typedef struct No
{
    Palavra *palavra;
    struct No *esq;
    struct No *dir;
    int altura;
} No;

typedef struct No *ArvBin;
\end{lstlisting}

Para a árvore binária não balanceada, guardamos uma palavra em cada nó, além dos nós esquerda e direita. A raiz é definida como um ponteiro de No.

\section{Árvores Binárias Balanceadas}\label{tadAVL}
Como já visto no curso, a árvore binária balanceada é definida da mesma forma que a não balanceada:
\begin{lstlisting}
typedef struct No *ArvAVL;
\end{lstlisting}

No entanto, as duas se diferem quanto ao balanceamento. Na árvore AVL, ela é constantemente balanceada ao inserir nós com Palavras.

\section{Árvores de Prefixo (trie)}\label{tadTrie}
\begin{lstlisting}
typedef struct NoTrie
{
    Palavra *palavra;
    struct NoTrie *filhos[TAM_TRIE];
} NoTrie;
\end{lstlisting}

A árvore trie foi implementada armazenando-se uma Palavra em cada nó, além de um vetor de nós de filhos do tamanho estipulado \texttt{TAM\_TRIE}, cujo valor é 36 (26 letras do alfabeto latino + 10 algarismos arábicos).


\section{Tabelas de Dispersão (tabela hash)}
\begin{lstlisting}
typedef struct tabelahash
{
    ArvAVL *hash[TAM_HASH];
    int *pesos;
} TabelaHash;
\end{lstlisting}
Por fim, temos a tabela hash, com o vetor da hash e o vetor de pesos. Em nossa implementação, optamos por utilizar árvore binárias balanceadas para guardar elementos, visto que é uma estrutura de armazenamento muito eficiente, como notado na elaboração do laboratório 11 da disciplina. \texttt{TAM\_HASH} foi definido como 100003, um número primo grande, escolhido após testes com arquivos grandes e pequenos, que é razoável para lidar com a grande faixa de tamanhos de arquivos proposta pelos autores.

\section{Funcionamento}
Após compilar o projeto, o usuário deve fornecer o número de buscas e os arquivos de entrada como argumento no terminal, nesta ordem, como explicado na Introdução. Com o programa em execução, será solicitado ao usuário informar qual estrutura de dados será utilizada para indexação dos arquivos. O usuário pode escolher quais quiser, uma por pedido, ou requisitar que todas sejam utilizadas de uma vez (em sequência). Após carregar os arquivos para a memória e realizar a busca randômica, será solicitado ao usuário que entre com uma palavra para ser buscada na(s) estrutura(s). O programa retornará os dados conforme a existência ou não da palavra no texto e, por fim, informará ao usuário os tempos de inserção, busca aleatória e busca do usuário.
