\chapter{Metodologia}\label{cap-metodologia}

% Descrever a metodologia dos testes, como variou o tamanho dos arquivos, quantos arquivos foram utilizados, descrição do computador em que foram feitos os testes.

Utilizando scripts de bash, pudemos automatizar os processos de testes, facilitando muito o levantamento de dados, possibilitando a obtenção de um maior espaço amostral e reduzindo os erros humanos. 

Os testes foram realizados executando 50 iterações do programa. Por exemplo, um carregamento de 10 arquivos com 500 buscas aleatórias seria repetido 50 vezes. Assim, garantimos que o espaço amostral seja satisfatório para a obtenção de dados pouco enviesados por \textit{outliers} --- neste caso, buscas e inserções muito lentas ou muito rápidas --- sobre a execução do programa. Foram feitas 100 buscas aleatórias em cada iteração, ao final tomando-se o tempo médio das 100 buscas para cada uma das 50 iterações, em que mais uma vez é tomada a média dos tempos, desta vez das 50 iterações.

Além disso, todos os testes foram efetuados em duas máquinas diferentes. A Máquina 1 conta com 16GB de RAM, swap de 2GB e um processador Intel i7-7700 com 8 núcleos rodando a 3.60GHz com um sistema operacional Linux Ubuntu 18.04.2 LTS; já a Máquina 2 possui 8GB de RAM, swap de 2GB, um processador AMD FX-6300 com 6 núcleos rodando a 3.80GHz com um sistema operacional Linux Mint 19.1 Cinnamon. Utilizaremos a Máquina 1 como referência nos testes de performance do Capítulo~\ref{cap-analise-resultado}, com os testes da Máquina 2 disponíveis no Apêndice~\ref{apendiceA}.

Os arquivos utilizados para testes variaram de 6b --- uma palavra --- a 1GB --- dados da Wikipedia ---, de 1 a 15 arquivos por testes. Arquivos com mais de 20MB usualmente demoram mais de uma hora para testar em listas encadeadas, portanto este foi limite estipulado para essa estrutura, com 1GB sendo o limite de todas as outras estruturas. É importante ressaltar que o indexador foi projetado para ler apenas o alfabeto latino de 26 caracteres e os 10 algarismos arábicos, principalmente por conta da necessidade de se estabelecer um número de nós filhos da árvore trie.


\begin{comment}

A Figura \ref{fig:graf} exemplifica o uso de uma figura gráfica no texto.

\begin{figure}[!htb]
\centering
%   \includegraphics[scale=0.90]{figuras/graf-exemplo.png}
  \caption{Exemplo de inserção de figura}
  \label{fig:graf}
\end{figure}
\end{comment}
