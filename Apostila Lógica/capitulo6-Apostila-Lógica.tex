\chapter{ARGUMENTOS NA LÓGICA DE PREDICADOS}

\section{Instanciação Universal (I.U.)}
Seja P(x) uma fórmula e \textit{a} um termo do universo em questão:
\[
\boxed{\frac{\forall x P(x)}{P(a)}}
\]

\noindent \textit{\textbf{Exemplo}}:

TODOS OS HOMENS SÃO PECADORES

TODOS OS PECADORES SERÃO PUNIDOS POR DEUS

SÓCRATES É HOMEM.


\noindent Mostre que o argumento acima é valido.

\section{GENERALIZAÇÃO EXISTENCIAL (G.E.)}
Seja P(x) uma fórmula e \textit{a} um termo do universo em questão:
\[
\boxed{\frac{P(a)}{\exists x P(x)}}
\]

\noindent \textit{\textbf{Exemplo}}:

SÓCRATES É FELIZ E FAMOSO

SE ALGUÉM É FAMOSO OU ELEGANTE, ENTÃO É RENOMADO

PORTANTO, EXISTE ALGUÉM FELIZ E RENOMADO

\noindent Mostre que o argumento acima é válido.

\section{INSTANCIAÇÃO EXISTENCIAL (I.E.)}

\[
\boxed{\frac{\exists x P(x)}{\exists P(a)}}
\]

\begin{itemize}
    \item O termo ``a'' não deve ocorrer nas premissas do argumento. Isto quer dizer que o termo ``a'' só deve aparecer no argumento por força da aplicação da regra I.E.
    \item Uma constante que tenha sido introduzida em um argumento por aplicação da regra I.E. não pode reaparecer, nesse argumento, por uma nova aplicação de I.E.
\end{itemize}

Por exemplo, considere os casos abaixo:
\begin{alignat*}{12}
    \text{a) } & \text{pr}       & \text{1 - } & \exists x P(x) &\\
               & \text{pr}       & \text{2 - } & Q(a) &\\
    \bullet\;  & \text{1, I.E. } & \text{3 - } & P(a) &\\
               & \text{2, 3, C } & \text{4 - } & P(a) \land Q(a) &\\
               & \text{4, G.E. } & \text{5 - } & \exists x (P(x)\ \land Q(x)) &\\
               & & & &\\ % Linha em branco
    \text{b) } & \text{pr}       & \text{1 - } & \exists x Q(x) &\\
               & \text{pr}       & \text{2 - } & \exists x \sim Q(x) &\\
               & \text{1, I.E. } & \text{3 - } & Q(a) &\\
    \bullet\;  & \text{2, I.E. } & \text{4 - } & \sim Q(a) &\\
               & \text{3, 4, C } & \text{5 - } & Q(a)\ \land \sim Q(a) &\\
               & \text{5, G.E. } & \text{6 - } & \exists x (Q(x)\ \land \sim Q(x)) &\\
\end{alignat*}
% \enlargethispage*{\baselineskip}
$\bullet$ INCORRETO
\pagebreak

\noindent \textit{\textbf{Exemplo}}:

ALGUNS ESTUDANTES SÃO DISCIPLINADOS

UM ESTUDANTE DISCIPLINADO RESPEITA SEU MESTRE

LOGO, ALGUNS ESTUDANTES RESPEITAM SEU MESTRE

\noindent Mostre que o argumento acima é valido.

\section{GENERALIZAÇÃO UNIVERSAL (G.U.)}
Seja P(x) uma fórmula e \textit{z} um termo arbitrário do universo em questão:

\[
\boxed{\frac{P(z)}{\forall x P(x)}}
\]

\begin{itemize}
    \item Não se deve aplicar a G.U. a constantes que ocorram nas premissas, pois estas se referem a particulares objetos do domínio.

    \item Não se deve aplicar a G.U a constantes introduzidas pela regra I.E., porque estas também se referem a particulares objetos do domínio.
\end{itemize}

\noindent \textit{\textbf{Exemplo}}:

TODO MUNDO É MARXISTA OU CAPITALISTA

NENHUM ÁRABE É MARXISTA

LOGO, TODOS OS ÁRABES SÃO CAPITALISTAS

\noindent Mostre que o argumento acima é válido.

\bigskip
\noindent \textbf{EXERCÍCIOS:}
\begin{enumerate}[label=\arabic*)]
    \item Use as regras de inferência para mostrar a validade dos argumentos abaixo:
    \begin{enumerate}[label=\alph*)]
        \item Todos os membros da Associação vivem na cidade. Quem é presidente  da sociedade é membro da Associação. Sra. Farias é presidente da Sociedade. Logo, Sra. Farias vive na cidade.
        \item Todas as linguagens de programação são importantes. Nenhuma linguagem de programação importante será dispensada. PROLOG é uma linguagem de programação. Logo, há linguagens de programação que não são dispensadas.
        \item Todas as vítimas são inocentes. Todos os inocentes serão absolvidos pela justiça. Maria é vítima, logo Maria será absolvida pela justiça.
        \item Paulo é estudioso e simpático. Se alguém é simpático ou inteligente, então é popular. Portanto, existe alguém estudioso e popular.
        \item Alguns soldados são heróis. Portanto, os heróis são valentes. Logo, alguns soldados são valentes.
    \end{enumerate}
\end{enumerate}
