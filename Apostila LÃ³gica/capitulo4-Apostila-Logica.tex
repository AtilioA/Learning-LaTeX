\chapter{Quantificadores}

\section{Uso intuitivo dos quantificadores}

Quantificadores são operadores que, em geral, transformam enunciados abertos em  enunciados fechados.
Como vimos na Gramática, os quantificadores formam enunciados(fechados) a partir de  letras argumentos e  de  enunciados.
Por exemplo, seja o seguinte enunciado aberto \textit{x gosta de programar}, que pode ser assim representado

\begin{align*}
    \textbf{gosta(x, programar)} \tag{1} \label{xGostaProgramar}
\end{align*}

Aqui, a letra x ocorre livre e assim sendo o enunciado \ref{xGostaProgramar} é um enunciado aberto.
Mas se prefixamos o quantificador existencial associado a letra argumento x, teremos

$$\exists x \colon gosta(x, programar)$$

\noindent que é um enunciado fechado e como tal verdadeiro ou falso.
A variável x que ocorre no enunciado \ref{xGostaProgramar} está associada a um universo de discurso.
No  exemplo acima, este universo pode ser, digamos, o conjunto dos alunos do curso de PROLOG.

De início, vamos apresentar, de forma intuitiva e natural, o uso dos quantificadores.
Para tanto, fixemos um certo conjunto, A, que tenha pelo menos um  elemento no  qual as variáveis vão tomar seus valores \textbf{--} i.e,

$$A = \left\{ 0, 2, 4, 6\right\}$$

Sejam ainda os seguintes enunciados:
\begin{enumerate}[label=\textbf{\alph*(x)} $\colon$]
    \setcounter{enumi}{15}
    \item x é número par
    \item x é múltiplo de 3
    \item x é divisor de 2
    \item x é maior ou igual a 15
    \item 2 é primo
\end{enumerate}

\noindent Consideremos o enunciado p(x); isto é,

\centerline{x é número par}

\noindent Este é um enunciado aberto.

Podemos a partir deste obter os seguintes enunciados fechados usando os quantificadores:

\begin{align*}
    \exists x\ & (\text{x é número par}) \\
    \forall x\ & (\text{x é número par}) \\
    \sim \exists x\ & (\text{x é número par}) \\
    \sim \forall x\ & (\text{x é número par})
\end{align*}

aqui a variável x toma valores no conjunto A.

\noindent Consideremos agora o enunciado t(x), i.e.:

\centerline{2 é primo}

\noindent Nele não ocorre a variável x, mas, apesar disso, admite-se prefixar a t(x) tanto $\exists$ quanto $\forall$, obtendo-se

\begin{align*}
    \exists x\ & (\text{2 é primo}) \\
    \forall x\ & (\text{2 é primo})
\end{align*}

\noindent Esses dois enunciados não só são considerados equivalentes entre si, como também são equivalentes a

\centerline{2 é primo}

Assim, se x não ocorre em p, então $\exists$x é equivalente a p.
Do mesmo modo, se x não ocorre em p, $\forall$x p é também equivalente ao próprio p.
Neste caso os quantificadores não desempenham qualquer função.

Considerando o  conjunto A do  exemplo anterior como universo de discurso, cabe fazer as seguintes afirmações:

\begin{enumerate}[label=(\roman*)]
    \item Todo elemento de A satisfaz a p(x).
    \item Existe pelo menos um elemento de A que satisfaz a q(x).
    \item Existe um só elemento de A que satisfaz a r(x).
    \item Não existe elemento de A que satisfaz a s(x).
\end{enumerate}

A primeira expressão é usualmente simbolizada assim:

$$\forall x\ (x \in A \to p(x))$$

Verifica-se, substituindo-se x pelos (nomes dos) elementos do universo A, que essa expressão é um enunciado verdadeiro.

Costuma-se, com frequência, abreviar o enunciado acima nada dizendo sobre o universo de discurso A, sempre que isto não causar equívocos.

\begin{align*}
    \forall x\ & p(x) \\
    \forall x\ & (\text{x é número par})
\end{align*}

Em resumo, as expressões acima podem ser assim simbolizadas:

\begin{enumerate}[label=(\roman*)]
    \item $\forall x\ (x \in A \to p(x))$

    (que assim pode ser lida: qualquer que seja x, x pertence a A,
    então p(x))
    \item $\exists x\ (x \in A \land q(x))$

    (que assim pode ser lida: existe x, tal que x pertence a A e q(x))
    \item $\exists!\ x\ (x \in A \land r(x))$

    (que assim pode ser lida: existe um único x, tal que x pertence a A e r(x))
    \item $\sim \exists x\ (x \in A \land s(x))$

    (que assim pode ser lida: não existe x, tal que x pertence a A e s(x))
\end{enumerate}

É importante não esquecer que a informação de que não existe ou de que existe apenas um elemento que satisfaz a uma propriedade p pode ser dada através do quantificador \textbf{existe no máximo um}, que é assim simbolizado: \underline{$\exists$} e é lido: existe no máximo um elemento.
Portanto, $\underline{\exists} x \colon p(x)$ é lido do seguinte modo: existe no máximo um elemento x que satisfaz à propriedade p.

\begin{exemplo}
    Para afirmar que existe no máximo um número par e primo, basta escrever:

    $$\underline{\exists}x\ (par(x) \land primo(x))$$
\end{exemplo}

\noindent O universo de discurso não indicado explicitamente foi até aqui o conjunto dos números naturais.

\noindent \textbf{Comentário:} O quantificador existencial ($\exists$) não afirma que apenas um único objeto do universo de discurso possua uma certa propriedade.
Assim, quando enunciamos

$$\exists x\ (aluno(x) \land programador(x))$$

\noindent pode ocorrer que mais de um aluno seja programador.
Mas, por vezes, estamos interessados em verificar se existe um único objeto dotado de uma certa propriedade p, i.e., simbolicamente,

$$\exists! x\ p(x)$$

De modo geral, podemos caracterizar esse quantificador da seguinte forma:

$$\exists! x\ p(x) \leftrightharpoons \exists x\ p(x) \land \underline{\exists}x\ p(x)$$

Dos exemplos acima, observamos que em se tratando de quantificadores, tanto os predicados envolvidos quanto o universo de discurso tem papel decisivo na determinação dos valores de verdade das fórmulas.
Vejamos o exemplo:
