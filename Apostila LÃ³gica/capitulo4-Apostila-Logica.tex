\chapter{Quantificadores}

\section{Uso intuitivo dos quantificadores}

Quantificadores são operadores que, em geral, transformam enunciados abertos em  enunciados fechados.
Como vimos na Gramática, os quantificadores formam enunciados(fechados) a partir de  letras argumentos e  de  enunciados.
Por exemplo, seja o seguinte enunciado aberto \textit{x gosta de programar}, que pode ser assim representado

\begin{align*}
    \textbf{gosta(x, programar)} \tag{1} \label{xGostaProgramar}
\end{align*}

Aqui, a letra x ocorre livre e assim sendo o enunciado \ref{xGostaProgramar} é um enunciado aberto.
Mas se prefixamos o quantificador existencial associado a letra argumento x, teremos

$$\exists x \colon gosta(x, programar)$$

\noindent que é um enunciado fechado e como tal verdadeiro ou falso.
A variável x que ocorre no enunciado \ref{xGostaProgramar} está associada a um universo de discurso.
No  exemplo acima, este universo pode ser, digamos, o conjunto dos alunos do curso de PROLOG.

De início, vamos apresentar, de forma intuitiva e natural, o uso dos quantificadores.
Para tanto, fixemos um certo conjunto, A, que tenha pelo menos um  elemento no  qual as variáveis vão tomar seus valores \textbf{--} i.e,

$$A = \left\{ 0, 2, 4, 6\right\}$$

Sejam ainda os seguintes enunciados:
\begin{enumerate}[label=\textbf{\alph*(x)} $\colon$]
    \setcounter{enumi}{15}
    \item x é número par
    \item x é múltiplo de 3
    \item x é divisor de 2
    \item x é maior ou igual a 15
    \item 2 é primo
\end{enumerate}

\noindent Consideremos o enunciado p(x); isto é,

\centerline{x é número par}

\noindent Este é um enunciado aberto.

Podemos a partir deste obter os seguintes enunciados fechados usando os quantificadores:

\begin{align*}
    \exists x\ & (\text{x é número par}) \\
    \forall x\ & (\text{x é número par}) \\
    \sim \exists x\ & (\text{x é número par}) \\
    \sim \forall x\ & (\text{x é número par})
\end{align*}

aqui a variável x toma valores no conjunto A.

\noindent Consideremos agora o enunciado t(x), i.e.:

\centerline{2 é primo}

\noindent Nele não ocorre a variável x, mas, apesar disso, admite-se prefixar a t(x) tanto $\exists$ quanto $\forall$, obtendo-se

\begin{align*}
    \exists x\ & (\text{2 é primo}) \\
    \forall x\ & (\text{2 é primo})
\end{align*}

\noindent Esses dois enunciados não só são considerados equivalentes entre si, como também são equivalentes a

\centerline{2 é primo}

Assim, se x não ocorre em p, então $\exists$x é equivalente a p.
Do mesmo modo, se x não ocorre em p, $\forall$x p é também equivalente ao próprio p.
Neste caso os quantificadores não desempenham qualquer função.

Considerando o  conjunto A do  exemplo anterior como universo de discurso, cabe fazer as seguintes afirmações:

\begin{enumerate}[label=(\roman*)]
    \item Todo elemento de A satisfaz a p(x).
    \item Existe pelo menos um elemento de A que satisfaz a q(x).
    \item Existe um só elemento de A que satisfaz a r(x).
    \item Não existe elemento de A que satisfaz a s(x).
\end{enumerate}

A primeira expressão é usualmente simbolizada assim:

$$\forall x\ (x \in A \to p(x))$$

Verifica-se, substituindo-se x pelos (nomes dos) elementos do universo A, que essa expressão é um enunciado verdadeiro.

Costuma-se, com frequência, abreviar o enunciado acima nada dizendo sobre o universo de discurso A, sempre que isto não causar equívocos.

\begin{align*}
    \forall x\ & p(x) \\
    \forall x\ & (\text{x é número par})
\end{align*}

Em resumo, as expressões acima podem ser assim simbolizadas:

\begin{enumerate}[label=(\roman*)]
    \item $\forall x\ (x \in A \to p(x))$

    (que assim pode ser lida: qualquer que seja x, x pertence a A,
    então p(x))
    \item $\exists x\ (x \in A \land q(x))$

    (que assim pode ser lida: existe x, tal que x pertence a A e q(x))
    \item $\exists!\ x\ (x \in A \land r(x))$

    (que assim pode ser lida: existe um único x, tal que x pertence a A e r(x))
    \item $\sim \exists x\ (x \in A \land s(x))$

    (que assim pode ser lida: não existe x, tal que x pertence a A e s(x))
\end{enumerate}

É importante não esquecer que a informação de que não existe ou de que existe apenas um elemento que satisfaz a uma propriedade p pode ser dada através do quantificador \textbf{existe no máximo um}, que é assim simbolizado: \underline{$\exists$} e é lido: existe no máximo um elemento.
Portanto, $\underline{\exists} x \colon p(x)$ é lido do seguinte modo: existe no máximo um elemento x que satisfaz à propriedade p.

\begin{exemplo}
    Para afirmar que existe no máximo um número par e primo, basta escrever:

    $$\underline{\exists}x\ (par(x) \land primo(x))$$
\end{exemplo}

\noindent O universo de discurso não indicado explicitamente foi até aqui o conjunto dos números naturais.

\noindent \textbf{Comentário:} O quantificador existencial ($\exists$) não afirma que apenas um único objeto do universo de discurso possua uma certa propriedade.
Assim, quando enunciamos

$$\exists x\ (aluno(x) \land programador(x))$$

\noindent pode ocorrer que mais de um aluno seja programador.
Mas, por vezes, estamos interessados em verificar se existe um único objeto dotado de uma certa propriedade p, i.e., simbolicamente,

$$\exists! x\ p(x)$$

De modo geral, podemos caracterizar esse quantificador da seguinte forma:

$$\exists! x\ p(x) \leftrightharpoons \exists x\ p(x) \land \underline{\exists}x\ p(x)$$

Dos exemplos acima, observamos que em se tratando de quantificadores, tanto os predicados envolvidos quanto o universo de discurso tem papel decisivo na determinação dos valores de verdade das fórmulas.
Vejamos o exemplo:

\begin{equation}\label{para_todo_pertence_fruta}
    \forall x\ (pertence(x, U) \to fruta (x)) \tag{1}
\end{equation}

\noindent onde $U = \{pêra, laranja, maçã, banana, uva\}$.
Levando em conta o universo e o predicado ser fruta, o enunciado (\ref{para_todo_pertence_fruta}) é verdadeiro.
Mas, se mudarmos o predicado \textit{ser fruta} pelo predicado \textit{ser legume}, o enunciado (\ref{para_todo_pertence_fruta}) torna-se falso.
Por outro lado, se o universo fosse o conjunto

$$U = \{pera, maçã, banana, uva, batata\},$$

\noindent o enunciado (\ref{para_todo_pertence_fruta}) seria falso, porém, o enunciado
$$\exists x (pertence(x, U) \land fruta(x))$$

\noindent é verdadeiro.

Estabeleceremos agora algumas das principais equivalências de enunciados quantificados.
Cabe aqui ressaltar que tais equivalências são muitas vezes utilizadas como formas alternativas de traduzir expressões da linguagem natural para a linguagem lógica.

\bigskip
Pode-se facilmente observar que:

\begin{enumerate}[label=(\roman*)]
    \item Todos os computadores funcionam

    equivale a

    \centerline{Não há computador que não funcione}

    simbolicamente

    $$\forall x\ (computador(x) \to funciona(x))$$

    equivale a

    $$\sim \exists x\ (computador(x)\ \land \sim funciona(x))$$

    \bigskip
    Também podemos observar que:

    \item Existe linguagem de programação declarativa

    que simbolicamente é equivalente a:

    $$\exists x\ (linguagem-programação(x) \land declarativa(x))$$

    equivale a

    $$\sim \forall x\ (linguagem-programação \to \sim declarativa(x))$$

    \bigskip
    Note-se que:
    \item Não existe máquina de fazer dinheiro

    equivale a

    \centerline{Toda máquina não faz dinheiro}

    simbolicamente

    $$\sim \exists x\ (máquina(x) \land faz(x, dinheiro))$$

    é equivalente a

    $$\forall x\ (máquina(x) \to \sim faz(x, dinheiro))$$

    \bigskip
    Finalmente:
    \item Nem toda máquina supera o homem

    equivale a

    \centerline{Existe máquina que não supera o homem}

    simbolicamente

    $$\sim \forall x\ (máquina(x) \to supera(x, homem))$$

    é equivalente a

    $$\exists x\ (máquina(x)\ \land \sim supera(x, homem))$$

\end{enumerate}

Em resumo, temos:

\begin{align*}
    \forall x\ \alpha (x) & \iff \sim \exists x \sim \alpha (x) \\
    \exists x\ \alpha (x) & \iff \sim \forall x \sim \alpha (x) \\
    \sim \forall x\ \alpha (x) & \iff \exists x \sim \alpha (x) \\
    \sim \exists x\ \alpha (x) & \iff \forall x \sim \alpha (x)
\end{align*}

Diferentemente do Cálculo dos Juntores, o  Cálculo de Predicados não possui, para toda sua extensão, um  processo efetivo, como as tabelas de verdade, para verificar a validade de certos enunciados.
Mais adiante vamos apresentar formas (obviamente, não efetivas) de verificar a validade dos enunciados do Cálculo de Predicados.

\noindent \textbf{Comentário:} os enunciados onde aparecem quantificadores são ditos enunciados quantificados.
O valor lógico de tais enunciados será determinado em função do universo de discurso associado a estes.


\section{Variáveis livres e ligadas}
Com frequência, em uma teoria ocorrem expressões (termos ou  enunciados) com variáveis, como $\forall x (x \in \mathbb{N}) \to  x z \in \mathbb{Z}$, $x < 2$ e $x + 3$.

Na  lista acima, a primeira e a segunda expressões são enunciados, ditos \textbf{enunciados abertos}.
Não lhe atribuímos um valor lógico.
A última expressão é um \textbf{termo}, dito termo aberto.
Não é um nome.

Os operadores são, como vimos, aplicados a  expressões abertas para formar, em  geral, expressões fechadas (termos ou enunciados).
Assim, uma expressão aberta torna-se fechada, caso tenha suas variáveis instanciadas ou se a elas aplicarmos os operadores devidos.
Assim, dizemos que na expressão

$$x < 2$$
a variável x é livre, uma vez que ela pode ser instanciada por qualquer elemento do domínio de discurso.
Mas cabe não esquecer-se de que alguns desses valores transformam o  enunciado aberto em um enunciado verdadeiro, enquanto que outros o converterão em um enunciado falso.
Por exemplo, $1 < 2$ e $3 < 2$, respectivamente.

Questões também relacionadas a variáveis livres e ligadas são os conceitos de escopo de um  operador, ocorrência de expressões, início de uma expressão, etc.
Tais conceitos não são fáceis de serem caracterizados.

Aqui faremos apenas um  estudo intuitivo e  elementar destas noções.

Sejam os operadores $\forall, \exists, \uptau\ \text{e} (\ )$ e as letras x, y, z com ou sem índices.
De forma genérica, os operadores são aplicados a expressões para formar novas expressões.
Assim, aplicando-se quantificadores a expressão

\begin{equation}\label{3y}
    x \in \mathbb{Z}\ \land y \in \mathbb{Z} \to x + y = z \tag{1}
\end{equation}

obtemos

\begin{equation}\label{3x}
    \exists z\ \forall x\ \forall y\ (x \in \mathbb{Z}\ \land y \in \mathbb{Z} \to x + y = z) \tag{2}
\end{equation}

Em (\ref{3x}) temos três ocorrências da letra x e em (\ref{3y}) duas ocorrências; temos ainda em (\ref{3y}) três ocorrências de y e duas ocorrências de z.
Observe-se ainda que x é a letra argumento do quantificador $\forall$; y é a letra argumento do quantificador $\forall$ e z é a letra argumento do quantificador $\exists$ em (\ref{3x}).

Em (\ref{3y}) todas as ocorrências de x são livres (i.e, não estão em  conexão com nenhum operador); em (\ref{3x}) as três ocorrências de x são ligadas.
Na expressão (\ref{3x}), o escopo da 1ª ocorrência do quantificador $\forall$ e $\forall y\ (x \in \mathbb{Z} \land y \in \mathbb{Z} \to x + y = z)$, enquanto que o escopo da 2ª ocorrência do quantificador $\forall$ é

$$x \in \mathbb{Z} \land y \in \mathbb{Z} \to x + y = z$$

Na fórmula
\begin{equation}\label{1z}
    \forall x\ \exists y\ (x + y = z) \tag{3}
\end{equation}
as duas ocorrências de x são ligadas e a única ocorrência de z é livre.

Na expressão (\ref{3x}) anterior, o escopo do operador $\exists$ é a expressão

$$\forall x\ \forall y\ (x \in \mathbb{Z} \land y \in \mathbb{Z} \to x + y = z)$$
enquanto que na expressão (\ref{1z}), o escopo do operador $\forall$ é a expressão

$$\exists y\ (x + y = z)$$

Na expressão (\ref{3x}) há duas ocorrências do quantificador $\forall$.
O escopo do quantificador $\forall$.
O escopo do quantificador $\forall$ com respeito a x é

$$\forall y\ (x \in \mathbb{Z}\ \land\ y \in \mathbb{Z}\ \to\ x + y = z)$$

Note-se que uma variável pode ocorrer livre e ligada em uma expressão.
Por exemplo:

\begin{equation}\label{x_livre_ligado}
    x \in \mathbb{Z}\ \to\ \exists y\ \forall x \colon x < y \tag{4}
\end{equation}
a primeira ocorrência de x em \ref{x_livre_ligado} é livre, mas a segunda e terceira são ligadas.

Toda ocorrência de uma variável ou é livre ou é ligada.
