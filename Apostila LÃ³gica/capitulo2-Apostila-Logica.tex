\chapter{Lógica e Linguagem}

\section{Partículas Lógicas}

\subsection{Uso intuitivo dos juntores}

Estamos aqui interessados em estabelecer regras para o uso de certas partículas lógicas, como: \textbf{não}, \textbf{e},  \textbf{ou}, \textbf{se... então}, \textbf{se e somente se}, \textbf{nem} e \textbf{nor}, denominadas juntores ou conectivos lógicos (ou conjunções lógicas).
Sua função, como vimos na gramática anteriormente apresentada, é formar enunciados a partir de enunciados.
Para mostrar o uso técnico de tais partículas, com frequência buscamos exemplos da linguagem corrente, aproveitando as convenções estabelecidas para esta linguagem.

É importante ressaltar, agora, que cada enunciado, no desenvolvimento de nosso trabalho, admitirá um, e um único valor lógico, isto é, será necessariamente verdadeiro ou falso e nunca ambos.

\begin{enumerate}[label=\textbf{(\arabic*)}]
    \item \textbf{O juntor não} (simbolicamente: $\sim$)

    Dado um enunciado, podemos formar um outro enunciado \textbf{--} denominado negação do primeiro \textbf{--} com o uso do juntor $\sim$, i.e., do negador.

    Apesar da linguagem corrente formar, mais comumente, a negação de um enunciado colocando o não junto ao verbo, adotaremos aqui o procedimento de antepor o negador ao enunciado.
    Por outro lado, enquanto que na linguagem corrente o não é advérbio, aqui ele atua como juntor.
    Considerando o enunciado:

    \begin{center}
        o plenário está cheio
    \end{center}
    sua negação será
    \begin{center}
        $\sim$(o plenário está cheio)
    \end{center}
    que se lê: \textbf{não e o caso que o plenário esteja cheio}.
    Assim, dado um enunciado qualquer p, pode-se formar o enunciado $\sim$p , dito negação de p.
    Se p for um enunciado verdadeiro, $\sim$p é falso, e se p for falso, então $\sim$p é verdadeiro.
    Isto pode ser descrito através das chamadas tabelas de valores lógicos ou tabelas de verdade da negação:

    \begin{center}
        \begin{tabular}{c c}
            p & $\sim$p \\ \hline
            V & F \\
            F & V
        \end{tabular}
    \end{center}

    onde V e F indicam, respectivamente, os valores lógicos: verdadeiro e falso, do enunciado.

    \item \textbf{O juntor e} (simbolicamente: $\land$)

    Dados dois enunciados, podemos obter um terceiro, dito conjunção dos dois primeiros, pela ação do juntor $\land$, i.e., o conjuntor.
    Assim, dados dois enunciados:

    \begin{center}
        Brasília é uma cidade

        e

        Brasilia é a capital do Brasil
    \end{center}

    podemos formar pela ação do conjuntor a conjunção:

    \begin{center}
        (Brasília é uma cidade) $\land$ (Brasília é a capital do Brasil)
    \end{center}

    Importa ter presente que o uso dos juntores, em Lógica, permite ligar enunciados mesmo sem qualquer tipo de vinculo significativo entre eles \textbf{--} como por exemplo:

    O café está amargo $\land$ Cláudia estuda música

    De forma análoga a linguagem corrente, um enunciado conjuntivo só é verdadeiro se os enunciados componentes forem simultaneamente verdadeiros; caso contrário, será falso.
    Disto decorre a seguinte tabela de verdade:

    \begin{center}
        \begin{tabular}{c c c}
        p & q & p $\land$ q \\ \hline
        V & V & V \\
        V & F & F \\
        F & V & F \\
        F & F & F
        \end{tabular}
    \end{center}


    \item \textbf{O juntor ou} (simbolicamente: $\lor$)

    O enunciado obtido a partir de dois enunciados dados, com o uso do juntor $\lor$, é chamado de disjunção desses dois enunciados.

    Sabemos que em linguagem corrente existem, pelo menos, dois usos distintos do juntor \textbf{ou} \textbf{--} o uso exclusivo e o uso não-exclusivo.
    Exemplos destes fatos podem ser vistos nos seguintes enunciados:

    \begin{enumerate}[label=(\arabic*)]
        \item João servirá a Marinha ou à Aeronáutica \label{joao-servira}
        \item Maria lecionará LISP ou PROLOG \label{maria-lecionara}
    \end{enumerate}

    Em \ref{joao-servira}, pretende-se que um fato exclua de todo o outro, e para tanto usa o \textbf{ou} exclusivo.

    Em \ref{maria-lecionara}, o \textbf{ou} foi usado no sentido não-exclusivo.
    O ou que usamos em Lógica é o não-exclusivo.
    Assim, a partir de dois enunciados p e q, forma-se o enunciado p $\lor$ q, dito disjunção de p e q.

    A disjunção é verdadeira quando, pelo menos, um dos enunciados for verdadeiro; caso contrário, é falsa.
    A tabela dos valores lógicos da disjunção é:

    \begin{center}
        \begin{tabular}{c c c}
            p & q & p $\lor$ q \\ \hline
            V & V & V \\
            V & F & V \\
            F & V & V \\
            F & F & F
        \end{tabular}
    \end{center}

    \pagebreak

    \item \textbf{O juntor nem} (simbolicamente: $\downarrow$)\footnote{O simbolo $\downarrow$ é chamado de símbolo de Sheffer.
    Ele apresenta um especial interesse para a Lógica e para a Computação, posto que todos os demais juntores podem ser definidos a partir desse. Cf. Quine, [Lógica].}

    A partir dos enunciados p e q, podemos formar o enunciado p $\downarrow$ q  (lê-se: nem p nem q), dito negação conjunta de p e q.
    Uma negação conjunta será verdadeira quando ambos os componentes forem falsos.
    Assim, tem-se a seguinte tabela de valores lógicos para a negação conjunta:

    \begin{center}
        \begin{tabular}{c c c}
            p & q & p $\downarrow$ q \\ \hline
            V & V & F \\
            V & F & F \\
            F & V & F \\
            F & F & V
        \end{tabular}
    \end{center}

    \item \textbf{O juntor se... então} (simbolicamente: $\to$)

    Caracterizaremos agora o uso do juntor $\to$.
    Se p e q são enunciados, p $\to$ q será dito condicional de p e q.
    No condicional $2 \in \mathbb{N} \to 2 \in \mathbb{Z}$, tem-se que $2 \in \mathbb{N}$ é o primeiro constituinte ou antecedente do condicional e $2 \in \mathbb{Z}$ é o segundo constituinte ou consequente do condicional.
    Para melhor compreendermos esse juntor, analisaremos o seguinte exemplo:

    \pagebreak

    Admitamos que o indivíduo A pergunta a B se uma certa afirmativa e válida.
    O indivíduo B, ao observar que A está lendo Knuth\footnotemark, responde:

    \begin{enumerate}[label=(\arabic*)]
        \item Se a afirmativa é de Knuth, então a afirmativa é válida. \label{knuth-valida}

        Analisamos sob que condições considerar-se-ia a resposta de B como verdadeira, ou falsa, na linguagem corrente.
        Usando $\to$, \ref{knuth-valida} é equivalente a:

        \item A afirmação e de Knuth $\to$ a afirmação é válida. \label{knuth-se-entao}
    \end{enumerate}

    O antecedente de \ref{knuth-se-entao}, p, é:

    \centerline{a afirmação é de Knuth}

    O consequente de \ref{knuth-se-entao}, q, é:

    \centerline{a afirmação é válida}

    Temos vários casos a considerar:

    % Knuth ofereceu 2,56 dólares e não 1 dólar (https://en.wikipedia.org/wiki/Knuth_reward_check)
    \footnotetext{Mesmo os melhores autores não estão salvos de publicar lapsos, sem consciência destes.
    Knuth ao escrever o seu livro \textbf{The Art of Computer Programming}, acreditando não conter erros, prometeu à comunidade científica presentear com 2,56 dólares aquele que detectasse alguma falha em tal publicação.
    Um erro foi encontrado, por ex., por Gaston Gonett, obrigando assim a Knuth pagar-lhe a quantia oferecida.
    Conta-se que Gaston Gonett não gastou o cheque de 2,56 dólares, mandando fazer com este um quadro, que se orgulha de ver pendurado em sua parede.}

    Admitamos que a afirmação não é de Knuth, e, daí, que p é falsa.
    Neste caso não se consideraria (normalmente) a resposta \ref{knuth-se-entao} de B como sendo falsa.
    B não assumiu incondicionalmente a responsabilidade de validade da afirmação.
    Disse sim que a afirmação é válida, se a afirmação é de Knuth.
    A sentença \ref{knuth-se-entao} é aqui admitida como verdadeira no sentido que a resposta (condicional) \ref{knuth-se-entao} de B não é considerada falsa.

    Suponhamos agora que a afirmação é realmente de Knuth, isto, é, que p é verdadeiro.
    \begin{enumerate}[label=\arabic*)]
        \item Se afirmação é, de fato, válida, isto é, se q é verdadeira, então \ref{knuth-se-entao} é normalmente considerada verdadeira.
        \item Se a afirmação, lamentavelmente, não é válida, isto é, que q é falsa, então \ref{knuth-se-entao} é normalmente considerada falsa.
    \end{enumerate}
    Assim, apenas num único caso (p verdadeira e q falsa), \ref{knuth-se-entao} é considerada verdadeira.

    De maneira geral, se os enunciados p e q são respectivamente o \textbf{antecedente} e o \textbf{consequente} de um certo condicional, este é considerado falso apenas quando \textbf{p} é verdadeiro e \textbf{q} é falso.
    Em todos os demais casos, o condicional é considerado verdadeiro.
    Em particular, se o antecedente é falso, o condicional é considerado verdadeiro, seja o consequente verdadeiro ou falso.
    Um enunciado para nós será aqui sempre considerado, de modo um tanto "ingênuo", verdadeiro ou falso.
    Numa outra linguagem: diremos que um enunciado pode ter (um de) dois valores lógicos, ou o valor \textbf{verdade}, simbolicamente \textbf{V}; ou o valor \textbf{falsidade}, simbolicamente \textbf{F}.

    O quadro abaixo resume a situação relativa a determinação do valor de um enunciado condicional p $\to$ q (em conformidade com a discussão acima), quando são dados os valores do seu antecedente p e de seu consequente q:

    \begin{center}
        \begin{tabular}{c c c}
            p & q & p $\to$ q \\ \hline
            V & V & V \\
            V & F & F \\
            F & V & V \\
            F & F & V
        \end{tabular}
    \end{center}

    No condicional p $\to$ q, p é dito \textbf{condição suficiente} a q e q é dito \textbf{condição necessária} a p.

    \item \textbf{O juntor se e somente se} (simbolicamente: $\iff$)

    Dados dois enunciados, podemos formar um terceiro, dito bicondicional dos dois primeiros pela ação do juntor $\iff$.
    Assim, p $\iff$ q será dito bicondicional de p e q.
    Um enunciado dessa forma será considerado verdadeiro se seus constituintes tiverem o mesmo valor lógico, isto é, se ambos forem verdadeiros ou se ambos forem falsos.
    Tem-se, então, a seguinte tabela de verdades para o bicondicional:

    \begin{center}
        \begin{tabular}{c c c}
            p & q & p $\iff$ q \\ \hline
            V & V & V \\
            V & F & F \\
            F & V & F \\
            F & F & V
        \end{tabular}
    \end{center}

    Impõe-se ter presente as duas seguintes condições:
    \begin{enumerate}[label=(\roman*)]
        \item O juntor $\iff$ pode ser definido mediante -> e $\land$.

        Assim, a fórmula \bm{$p \iff q$} equivale à fórmula \bm{$(p \to q) \land (q \to p)$}

        \item Em \bm{$p \to q$}, \textbf{p} é dito condição necessária e suficiente de \textbf{q}; e \textbf{q} é dito condição necessária e suficiente de \textbf{p}.
    \end{enumerate}

    \item \textbf{O juntor nor} (simbolicamente: $\uparrow$)

    \resizebox{\textwidth}{!}{A partir dos enunciados \textbf{p} e \textbf{q}, podemos formar o enunciado p $\uparrow$ q} (lê-se: não p ou não q), dito negação disjuntiva de p e q.
    Uma negação disjuntiva será falsa apenas quando seus componentes forem verdadeiros.
    Assim, tem-se a seguinte tabela de valores lógicos para a negação disjuntiva:

    \begin{center}
        \begin{tabular}{c c c}
            p & q & p $\uparrow$ q \\ \hline
            V & V & F \\
            V & F & V \\
            F & V & V \\
            F & F & V
        \end{tabular}
    \end{center}

    \noindent \textbf{Comentário:} Na linguagem corrente, algumas vezes, utilizamos o \textbf{nem} numa acepção distinta da apresentada aqui, como por exemplo na proposição:

    \centerline{Nem toda máquina é eficiente.}

\end{enumerate}
