\documentclass[
	% --- opções da classe memoir ---
	12pt,				% Tamanho da fonte
	openright,			% Capítulos começam em pág ímpar (insere página vazia caso preciso)
	twoside,			% Para impressão em recto e verso. Oposto a oneside
	a4paper,			% Tamanho do papel.
	% -- opções do pacote babel --
	english,			% Idioma adonal para hifenização
	french,				% Idioma adicional para hifenização
	spanish,			% Idioma adicional para hifenização
	brazil,				% O último idioma é o principal do documento
	]{abntex2}


% ---
% Pacotes
% ---
\usepackage{lmodern}			% Usa a fonte Latin Modern
\usepackage[T1]{fontenc}		% Seleção de códigos de fonte.
\usepackage[utf8]{inputenc}		% Codificação do documento (conversão automática dos acentos)
\usepackage{indentfirst}		% Indenta o primeiro parágrafo de cada seção.
\usepackage{color}				% Controle das cores
\usepackage{graphicx}			% Inclusão de gráficos
\usepackage{microtype} 			% Para melhorias de justificação
\usepackage{float}
% \usepackage{mathabx}
\usepackage[ampersand]{easylist}
\usepackage{enumitem}
\usepackage{titlesec}
\usepackage{caption}
\usepackage{subcaption}
\usepackage{multicol}
\usepackage{multirow}
\usepackage{lipsum}
\usepackage{amsfonts}
\usepackage{siunitx}
\usepackage{amsmath}

\usepackage[brazilian,hyperpageref]{backref}
\usepackage[alf]{abntex2cite}

\setlist[enumerate]{itemsep=1pt}

%%%%%%%%%%%%%%%%%%%%%%%%%%%%%%%%%%%%%%%%%%%%%%%%%%
%% COLOR DEFINITIONS
%%%%%%%%%%%%%%%%%%%%%%%%%%%%%%%%%%%%%%%%%%%%%%%%%%
\usepackage[svgnames]{xcolor}
%%%%%%%%%%%%%%%%%%%%%%%%%%%%%%%%%%%%%%%%%%%%%%%%%%
\definecolor{MyColor1}{rgb}{0.2,0.4,0.6} %mix personal color
\newcommand{\textb}{\color{Black} \usefont{OT1}{lmss}{m}{n}}
\newcommand{\blue}{\color{MyColor1} \usefont{OT1}{lmss}{m}{n}}
\newcommand{\blueb}{\color{MyColor1} \usefont{OT1}{lmss}{b}{n}}
\newcommand{\red}{\color{LightCoral} \usefont{OT1}{lmss}{m}{n}}

\renewcommand\thesection{\arabic{section}.} %define sections numbering
\renewcommand\thesubsection{\thesection\arabic{subsection}} %subsec.num.

\newcommand{\mysection}{
\titleformat{\section} [runin] {\usefont{OT1}{lmss}{b}{n}\color{MyColor1}} 
{\thesection} {3pt} {} } 

\renewcommand{\theequation}{\thesection\arabic{equation}}

\title{\blue Estatística Básica \\
\blueb Lista de Exercícios 2}
\author{Atílio Antônio Dadalto\\Tiago da Cruz Santos}

\date{2019}
% \\\{(?=[A-Z])|\\\{(?=[0-9])
%%%%%%%%%%%%%%%%%%%%%%%%%%%%%%%%%%%%%%%%%%%%%%%%%%

\begin{document}
% Retira espaço extra obsoleto entre as frases.
\frenchspacing

\textual

\maketitle
\textual
\newpage
% \imprimirfolhaderosto

\section{Solução da Questão 1}
Podemos definir o espaço amostral da seguinte forma:

$\Omega = \{(A, A), (B, B), (A, C, C), (B, C, C), (A, C, B, A),\\
(A, C, B, B), (B, C, A, B), (B, C, A, A)\}$.

\section{Solução da Questão 2}
\begin{enumerate}[label=\alph*)]
    \item $\Omega = \{(1, 1), (1, 2), ..., (1, 6), (2, 1), (2, 2), ..., (2, 6), ..., (6, 6)\}$;
    \item $\Omega = \{0, 1, 2, ..., M\}$, com M sendo o número máximo de peças defeituosas;
    \item Se M = sexo masculino e F = sexo feminino:
    
    $\Omega = \{(M, M, M), (M, M, F), (M, F, M), (F, M, M),\\(M, F, F), (F, M, F), (F, F, M), (F, F, F)\}$;
    \item $\Omega = \{(N, N, ..., N), (S, N, ..., N), (S, S, ..., N), ..., (S, S, ..., S)\}$;
    \item $\Omega = \{t \mid t \in \mathbb{R}, t \geq 0\}$;
    \item $\Omega = \{3, 4, 5, ..., 10\}$, uma vez que o mínimo de fichas é 3 e o máximo é 10;
    \item $\Omega = \{1, 2, 3, ...\}$;
    \item $\Omega = \{\ang{0}, \ang{6}, \ang{12}, ..., \ang{360}\}$;
    \item Usando ``solteira'' = so, ``casada'' = ca, amasiada = am, ``divorciada'' = di e ``viúva'' = vi:
    
    $\Omega = \{(A, \text{so}), (A, \text{ca}), (A, \text{am}), (A, \text{di}), (A, \text{vi}),\\(B, \text{so}), (B, \text{ca}), ..., (B, \text{vi}), ..., (C, \text{vi}), ..., (D, \text{vi})\}$.
\end{enumerate}


\section{Solução da Questão 3}
Duas moedas são lançadas, logo $\Omega = \{CC, CR, RC, RR\}$ sendo $C$ = cara e $R$ = coroa.

\begin{enumerate}[label=\alph*)]
    \item $A = \{CC, CR, RC\}$.
    \item $A = \{CC\}$.
    \item $A = \{CR, RC, RR\}$.
\end{enumerate}

\section{Solução da Questão 4}
\begin{enumerate}[label=\alph*)]
    \item $A\cap B^c$.
    \item $(A\cap B^c) \cup (A^c \cap B)$.
    \item $A^c \cap B^c$.
\end{enumerate}

\section{Solução da Questão 5}
\begin{enumerate}[label=\alph*)]
    \item Para expressar os resultados possíveis, podemos determinar o espaço amostral do experimento.
    Isto é, com P = retirada de bola preta e V = retirada de bola vermelha, temos:
    
    $\Omega = \{(PP), (PV), (VP), (VV)\}$.

    Calculando as probabilidades, considerando que não há reposição: 
    $$P(PP) = \frac{3}{8}\cdot\frac{2}{7} = \frac{6}{56} \approx 0.107.$$
    $$P(PV) = \frac{3}{8}\cdot\frac{5}{7} = \frac{15}{56} \approx 0.267.$$
    $$P(VP) = \frac{5}{8}\cdot\frac{3}{7} = \frac{15}{56} \approx 0.267.$$
    $$P(VV) = \frac{5}{8}\cdot\frac{4}{7} = \frac{20}{56} \approx 0.357.$$
    
    \item Com reposição, temos:
    $$P(PP) = \frac{3}{8}\cdot\frac{3}{8} = \frac{9}{64} \approx 0.140.$$
    $$P(PV) = \frac{3}{8}\cdot\frac{5}{8} = \frac{15}{64} \approx 0.234.$$
    $$P(VP) = \frac{5}{8}\cdot\frac{3}{8} = \frac{15}{64} \approx 0.234.$$
    $$P(VV) = \frac{5}{8}\cdot\frac{5}{8} = \frac{25}{64} \approx 0.390.$$
    
\end{enumerate}

\section{Solução da Questão 6}

\begin{enumerate}[label=\alph*)]
    \item Para bola preta na primeira e segunda extrações, temos que, sem reposição,
    $$P(PP) = \frac{3}{8}\cdot\frac{2}{7} = \frac{6}{56} \approx 0.107.$$
    Com reposição, 
    $$P(PP) = \frac{3}{8}\cdot\frac{3}{8} = \frac{9}{64} \approx 0.140.$$
    
    \item Para bola preta na segunda extração, temos $P(PP) + P(VP)$, isto é:
    
    Sem reposição:
    $$\frac{6}{56} + \frac{15}{56} = \frac{21}{56} = 0.375.$$
    Com reposição:
    $$\frac{9}{64} + \frac{15}{64} = \frac{24}{64} = 0.375.$$
    
    \item Para bola vermelha na primeira extração, temos $P(VP) + P(VV)$, isto é:
    
    Sem reposição:
    $$\frac{5}{8}\cdot \frac{4}{7} + \frac{5}{8}\cdot \frac{3}{7} = \frac{35}{56} = 0.625$$
    Com reposição:
    $$\frac{5}{8}\cdot \frac{5}{8} + \frac{5}{8}\cdot \frac{3}{8} = \frac{40}{64} = 0.625$$
    
\end{enumerate}


\section{Solução da Questão 7}
Uma vez que os eventos são independentes, $P(A \cap B) = P(A)\cdot P(B)$. Se ambos estão tentando ao mesmo tempo, independentemente, temos que a probabilidade do problema ser resolvido é igual a $P(A \cup B) = P(A) + P(B) - P(A \cap B) = \frac{2}{3} + \frac{3}{4} - (\frac{2}{3} \cdot \frac{3}{4}) = \frac{8}{12} + \frac{9}{12} - \frac{6}{12} = \frac{11}{12}$.

\section{Solução da Questão 8}
\begin{enumerate}[label=\alph*)]
    \item $\frac{2800 + 7000}{15800} \approx 0.62.$
    \item $\frac{800 + 2500}{15800} \approx 0.21.$
    \item $\frac{1800}{15800} \approx 0.11.$
    \item $\frac{800}{2800} \approx 0.29.$
\end{enumerate}

\section{Solução da Questão 9}
\begin{enumerate}[label=\alph*)]
    \item $\frac{8300}{15800}\cdot\frac{8300}{15800} \approx 0.28.$
    \item $\frac{2800}{15800}\cdot\frac{2000}{15800} \approx 0.02.$
    \item $\frac{13000}{15800}\cdot\frac{13000}{15800} \approx 0.68.$
\end{enumerate}
\section{Solução da Questão 10}
\begin{enumerate}[label=\alph*)]
    \item $\frac{8300}{15800}\cdot\frac{8299}{15799} \approx 0.28.$
    \item $\frac{2800}{15800}\cdot\frac{2000}{15799} \approx 0.02.$
    \item $\frac{13000}{15800}\cdot\frac{12999}{15799} \approx 0.68.$
\end{enumerate}

\section{Solução da Questão 11}
Temos os seguintes caminhos possíveis entre os dois pontos:

$A = (1-4)$

$B = (2-5)$

$C = (1-3-5)$

$D = (2-3-4)$

Se algum desses caminhos puder ser utilizado, o sistema estará funcionando. Portanto, a probabilidade de funcionamento do sistema é $P(S) = P(A \cup B \cup C \cup D)$.

Seja $c_i$ o evento do $i$-ésimo componente funcionar, $i = \{1, 2, 3, 4, 5\}$.

$P(A) = P(c_1 \cap c_4)$

$P(B) = P(c_2 \cap c_5)$

$P(C) = P(c_1 \cap c_3 \cap c_5)$

$P(D) = P(c_2 \cap c_3 \cap c_4)$

$P(S) = P((c_1 \cap c_4) \cup (c_2 \cap c_5) \cup (c_1 \cap c_3 \cap c_5) \cup (c_2 \cap c_3 \cap c_4))$

% $P(S) = P(c_1 \cap c_4) + P(c_2 \cap c_5) + P(c_1 \cap c_3 \cap c_5) + P(c_2 \cap c_3 \cap c_4) - P(c_1\cap c_2\cap c_4\cap c_5) - P(c_1\cap c_3\cap c_4\cap c_5) - P(c_1\cap c_2\cap c_3\cap c_4) \text{da interseção do já errei}$
\vspace{0.5cm}
Sabemos que

\begin{align*}
P(A \cup B \cup C \cup D) &= P(A) + P(B) + P(C) + P(D) - P(A \cap B)\\
&- P(A \cap C) - P(A \cap D)- P(B \cap C) - P(B \cap D) - P(C \cap D)\\
&+ P(A \cap B \cap C) + P(A \cap B \cap D) + P(A \cap C \cap D)\\
&+ P(B \cap C \cap D) - P(A \cap B \cap C \cap D).
\end{align*}

Como os componentes funcionam de forma independente,
\begin{align*}
P(S) &= P(A) + P(B) + P(C) + P(D) - P(A) \cdot P(B)\\
&- P(A) \cdot P(C) - P(A) \cdot P(D) - P(B) \cdot P(C) - P(B) \cdot P(D) - P(C) \cdot P(D)\\
&+ P(A) \cdot P(B) \cdot P(C) + P(A) \cdot P(B) \cdot P(D) + P(A) \cdot P(C) \cdot P(D)\\
&+ P(B) \cdot P(C) \cdot P(D) - P(A) \cdot P(B) \cdot P(C) \cdot P(D)
\end{align*}


% Ignorar isso
% \begin{align*}
% % P(S) &= P(c_1 \cap c_4 \cup c_2 \cap c_5 \cup c_1 \cap c_3 \cap c_5 \cup c_2 \cap c_3 \cap c_4)\\
% % &= P(c_1 \cap c_4) + P(c_2 \cap c_5) + P(c_1 \cap c_3 \cap c_5) + P(c_2 \cap c_3 \cap c_4)\\
% % &- P(c_1 \cap c_4 \cap c_2 \cap c_5) - P(c_1 \cap c_4 \cap c_1 \cap c_3 \cap c_5)\\
% % &- P(c_1 \cap c_4 \cap c_2 \cap c_3 \cap c_4)- P(c_2 \cap c_5 \cap c_1 \cap c_3 \cap c_5)\\
% % &- P(c_2 \cap c_5 \cap c_2 \cap c_3 \cap c_4) - P(c_1 \cap c_3 \cap c_5 \cap c_2 \cap c_3 \cap c_4)\\
% % &+ P(c_1 \cap c_4 \cap c_2 \cap c_5 \cap c_1 \cap c_3 \cap c_5)\\
% % &+ P(c_1 \cap c_4 \cap c_2 \cap c_5 \cap c_2 \cap c_3 \cap c_4)\\
% % &+ P(c_1 \cap c_4 \cap c_1 \cap c_3 \cap c_5 \cap c_2 \cap c_3 \cap c_4)\\
% % &+ P(c_2 \cap c_5 \cap c_1 \cap c_3 \cap c_5 \cap c_2 \cap c_3 \cap c_4)\\
% % &- P(c_1 \cap c_4 \cap c_2 \cap c_5 \cap c_1 \cap c_3 \cap c_5 \cap c_2 \cap c_3 \cap c_4).
% \end{align*}

% &= P(c_1 \cap c_4) + P(c_2 \cap c_5) + P(c_1 \cap c_3 \cap c_5) + P(c_2 \cap c_3 \cap c_4)\\
% &- P(c_1 \cap c_4 \cap c_2 \cap c_5) - P(c_4 \cap c_1 \cap c_3 \cap c_5)\\
% &- P(c_1 \cap c_4 \cap c_2 \cap c_3) - P(c_2 \cap c_5 \cap c_1 \cap c_3)\\
% &- P(c_5 \cap c_2 \cap c_3 \cap c_4) + 2P(c_1 \cap c_5 \cap c_2 \cap c_3 \cap c_4)\\.

% Então

% \begin{align*}
% P(S) &= P(c_1 \cap c_4 \cup c_2 \cap c_5 \cup c_2 \cap c_5 \cup c_2 \cap c_3 \cap c_4)\\
% &= P(c_1 \cap c_4) + P(c_2 \cap c_5) + P(c_2 \cap c_5) + P(c_2 \cap c_3 \cap c_4)\\
% &- P(c_1 \cap c_4 \cap c_2 \cap c_5) - P(c_1 \cap c_4 \cap c_2 \cap c_5)\\
% &- P(c_1 \cap c_4 \cap c_2 \cap c_3 \cap c_4)- P(c_2 \cap c_5 \cap c_2 \cap c_5)\\
% &- P(c_2 \cap c_5 \cap c_2 \cap c_3 \cap c_4) - P(c_2 \cap c_5 \cap c_2 \cap c_3 \cap c_4)\\
% &+ P(c_1 \cap c_4 \cap c_2 \cap c_5 \cap c_2 \cap c_5)\\
% &+ P(c_1 \cap c_4 \cap c_2 \cap c_5 \cap c_2 \cap c_3 \cap c_4)\\
% &+ P(c_1 \cap c_4 \cap c_2 \cap c_5 \cap c_2 \cap c_3 \cap c_4)\\
% &+ P(c_2 \cap c_5 \cap c_2 \cap c_5 \cap c_2 \cap c_3 \cap c_4)\\
% &- P(c_1 \cap c_4 \cap c_2 \cap c_5 \cap c_2 \cap c_5 \cap c_2 \cap c_3 \cap c_4).
% \end{align*}

% Simplificando,
% \begin{align*}
%     P(S) &= P(c_2 \cap c_5) + P(c_2 \cap c_3 \cap c_4)\\
%     &- P(c_1 \cap c_4 \cap c_2 \cap c_5) -  P(c_1 \cap c_4 \cap c_2 \cap c_3)\\
%     &- P(c_2 \cap c_5 \cap c_3 \cap c_4) + P(c_1 \cap c_5 \cap c_2 \cap c_3 \cap c_4).
% \end{align*}

% Note que os conjuntos formados por 1, 4 e por 2, 5 são os únicos conectados em paralelo. Portanto $P(c_1 \cap c_4 \cap c_2 \cap c_5) = P(c_1 \cap c_4) * P(c_2 \cap c_5)$.

\section{Solução da Questão 12}
A = Peça ser da máquina A

B = Peça ser da máquina B

C = Peça ser da máquina C

D = Peça ser defeituosa

\begin{align*}
    P(A) &= 0.25 & P(D|A) = 0.05\\
    P(B) &= 0.35 & P(D|B) = 0.04\\
    P(C) &= 0.4 & P(D|C) = 0.02\\
\end{align*}

    $$P(A|D) = \frac{P(A\cap D)}{P(D)} = \frac{P(D|A)\cdot P(A)}{P(D)}$$
    $$P(B|D) = \frac{P(B\cap D)}{P(D)} = \frac{P(D|B)\cdot P(A)}{P(D)}$$
    $$P(C|D) = \frac{P(C\cap D)}{P(D)} = \frac{P(D|C)\cdot P(A)}{P(D)}$$

Pelo teorema da probabilidade total (assumindo que as produções das máquinas formem uma partição do espaço amostral):
\begin{align*}
    P(D) &= P(D|A)\cdot P(A) + P(D|B)\cdot P(B) + P(D|C)\cdot P(C)\\
    P(D) &= 0.05\cdot0.25 + 0.04\cdot0.35 + 0.02\cdot0.4 = 0.0345.
\end{align*}

$$P(A|D) = \frac{0.05\cdot0.25}{0.0345} \approx 0.362$$
$$P(B|D) = \frac{0.04\cdot0.35}{0.0345} \approx 0.405$$
$$P(C|D) = \frac{0.02\cdot0.4}{0.0345} \approx 0.231$$

\section{Solução da Questão 13}
Temos que:

$A_1 =$ Evento de uma pessoa da categoria 1 sofrer acidente.

$A_2 =$ Evento de uma pessoa da categoria 2 sofrer acidente.

$C_1 =$ Evento de uma pessoa ser da categoria 1.

$C_2 =$ Evento de uma pessoa ser da categoria 2.

e

$P(A_1) = 0.1.$

$P(A_2) = 0.05.$

$P(C_1) = 0.2.$

$P(C_2) = 0.8.$

\begin{enumerate}[label=\alph*)]
    \item \label{PAq13} Pelo teorema da probabilidade total (uma vez que $C_1$ e $C_2$ são eventos disjuntos e, unidos, formam todo o espaço amostral), a probabilidade de um novo cliente sofrer acidente no primeiro ano é igual a $P(A_1) \cdot P(C_1) + P(A_2) \cdot P(C_2) = 0.1 \cdot 0.2 + 0.05 \cdot 0.8 = 0.06.$
    
    \item Queremos calcular $P(C_1|A_1)$:
    
    $$P(C_1|A_1) = \frac{P(A_1|C_1) \cdot P(C_1)}{P(A)}$$
    % $$P(A) = 0.1 \cdot 0.2 + 0.05 \cdot 0.8 = 0.06$$
    Por \ref{PAq13}, $P(A) = 0.06$. Como os eventos $A_i$ são independentes dos eventos $C_i, i = \{1, 2\}$,
    $$P(C_1|A_1) = \frac{0.1 \cdot 0.2}{0.06} = \frac{1}{3}.$$
\end{enumerate}

\section{Solução da Questão 14}

Definindo nossos eventos:

E = Amigo esquecer de enviar a carta;

C = Correio extraviar a carta;

N = Carteiro não entregar a carta.

Sabemos que:
\begin{align*}
    P(E) &= 0.1\\
    P(C|E^c) &= 0.1\\
    P(N |\ C^c\cap E^c) &= 0.1
\end{align*}

Desejamos calcular
$$P(E|N) = \frac{P(N|E)\cdot P(E)}{P(N)}$$
em que
$$P(N) = P(N|E)\cdot P(E) + P(N|E^c)\cdot P(E^c)$$

Sabemos que a probabilidade de o carteiro não entregar a carta dado que o amigo esqueceu de enviá-la é 1. Logo:
\begin{align*}
    P(N) &= 1\cdot 0.1 + P(N|E^c)\cdot P(E^c)\\
    P(N) &= 0.1 + P(N|E^c) * 0.9\\
    P(N|E^c) &= P(N\cap C|E^c) + P(N\cap C^c|E^c)\\
\end{align*}
Teremos que calcular estes dois últimos termos separadamente:
\begin{align*}
    P(N\cap C^c|E^c) = \frac{P(N\cap C^c\cap E^c)}{P(E^c)} &= \frac{P(N\cap C^c\cap E^c)\cdot P(C^c\cap E^c)}{P(E^c)\cdot P(C^c\cap E^c)}\\
    &= P(N|C^c\cap E^c)\cdot P(C^c|E^c) = 0.1\cdot 0.9 = 0.09\\\\
    P(N\cap C|E^c) = \frac{P(N\cap C\cap E^c)}{P(E^c)} &=  \frac{P(N\cap C\cap E^c)\cdot P(C\cap E^c)}{P(E^c)\cdot P(C\cap E^c)}\\
    &= P(N|C\cap E^c)\cdot P(C|E^c) = 1\cdot 0.1 = 0.1
\end{align*}

Voltando para P(N):
\begin{align*}
    P(N) &= 0.1 + (0.09 + 0.1)\cdot 0.9\\
    P(N) &= 0.271\\
    P(E|N) &= \frac{0.1}{0.271} \approx 0.369.
\end{align*}
Essa é a probabilidade de o amigo ter esquecido de enviar a carta.

\end{document}