\begin{apendicesenv}

\chapter{Planos de testes}\label{apendiceA}

Este apêndice serve como repositório para os planos testes das funções mais importantes implementadas no projeto. Algumas funções assumem certos valores de entrada, de forma a afunilar as possíveis entradas, dado que os valores a serem recebidos podem variar enormemente e que o projeto conta com muitas funções; no entanto, nos contemos a assumir entradas de maneira muito razoável, e, além disso, todos os casos básicos foram tratados devidamente dentro das funções.


Para melhor exemplificação e legibilidade dos testes das funções abaixo, usaremos a lista de dados de entrada ilustrada a seguir:

\begin{lstlisting}
listaRobos = [
    ('robo3', 1, (7, 7), 3), ('robo4', 2, (7, 5), 2),
    ('robo3', 3, (5, 4), 3), ('robo3', 4, (8, 1), 4),
    ('robo4', 5, (4, 5), 3), ('robo5', 6, (7, 7), 4)
]
\end{lstlisting}

\section{\texttt{distancia\_total()}}
A função calcula a distância total entre uma lista de pontos somando a distância entre um ponto e o próximo até o fim da lista.

\begin{center}
    \begin{tabular}{|c|c|}
        \cline{1-2}
        lista & resultado esperado \\ \hline
        [] & None \\ \hline  
        lista com um ponto & None \\ \hline  
        [p1, p2, p3] & distância de p1 até p2 + distância de p2 até p3 \\ \hline
    \end{tabular}
\end{center}

Definindo entradas:

\begin{center}
    \begin{tabular}{|c|c|c|}
        \cline{1-3}
        lista & resultado esperado & resultado obtido  \\ \hline
        [] & None & None \\ \hline
        [(4,4)] & None & None \\ \hline
        [(0,0),(3,4),(5,6)] & 7.82842712474619 & 7.82842712474619 \\ \hline

    \end{tabular}
\end{center}

\section{\texttt{distancia\_total\_robo()}}
A função calcula a distância total percorrida por um robô, ou seja, a soma das distâncias de todos os pontos percorridos por ele.

\begin{center}
    \begin{tabular}{|c|c|}
        \cline{1-2}
        lista & resultado esperado \\ \hline
        [] & None \\ \hline  
        [(a,(p1)),(a,(p2))], 'a' & distância de (0,0) até p1 + distância de p1 até p2 \\ \hline
    \end{tabular}
\end{center}


Definindo entradas:

\begin{center}
    \begin{tabular}{|c|c|c|}
        \cline{1-3}
        lista & resultado esperado & resultado obtido  \\ \hline
        [] & None & None \\ \hline
        [('robo1',2,(4,4),5)], 'robo2' & None & None \\ \hline
        listaRobos, 'robo3' & 17.74768689919494 & 17.74768689919494 \\ \hline

    \end{tabular}
\end{center} 


\section{\texttt{indices\_maximos()}}
A função recebe uma lista numérica e retorna uma lista com os índices de todas as ocorrências do valor máximo na lista.

\begin{center}
    \begin{tabular}{|c|c|}
        \cline{1-2}
        lista & resultado esperado \\ \hline
        [] & [] \\ \hline  
        [a, b, c, d] & índices das ocorrências do maior elemento da lista  \\ \hline
    \end{tabular}
\end{center}

Definindo entradas:

\begin{center}
    \begin{tabular}{|c|c|c|}
        \cline{1-3}
        lista & resultado esperado & resultado obtido  \\ \hline
        [] & [] & [] \\ \hline
        [5] & [0] & [0] \\ \hline
        [1,2,3,10,1,10] & [3,5] & [3,5] \\ \hline

    \end{tabular}
\end{center} 


\section{\texttt{imprime\_robos\_mais\_distantes()}}
A função recebe uma lista de robôs e imprime os robôs que encontram-se mais distantes da origem, seus percursos e tempos de percurso
\begin{center}
    \begin{tabular}{|c|c|}
        \cline{1-2}
        lista & resultado esperado \\ \hline
        [] & None \\ \hline  
        [('b',(10,10),1)] & ['b'],[(0,0),(10,10)],[1] \\ \hline
        [('a',(5,4),1),('b',(10,10),1)] & ['b'],[(0,0),(10,10)],[1]  \\ \hline
    \end{tabular}
\end{center}

Definindo entradas:

\begin{center}
    \begin{tabular}{|c|c|c|}
        \cline{1-3}
        lista & resultado esperado & resultado obtido  \\ \hline
        [] & None & None \\ \hline
        [('robo',1,(10,10),3)] & [('robo',((0,0),(10,10)),1] & [('robo',((0,0),(10,10)),1] \\ \hline
        [('robo',1,(10,10),3),('robo2',1,(2,1),3)] & [('robo',((0,0),(10,10)),3] & [('robo',((0,0),(10,10)),3]  \\ \hline

    \end{tabular}
\end{center}




\section{\texttt{caminhos\_robos\_crescente()}}
A função retorna uma lista de tuplas para cada robô, ordenadas em ordem crescente pela distância total percorrida pelos robôs.

\begin{center}
    \begin{tabular}{|c|c|}
        \cline{1-2}
        Entradas & resultado esperado \\ \hline
        [] &  None\\ \hline
        [(a),(b)] & soma do caminho percorrido a, soma do caminho b  \\ \hline
    \end{tabular}
\end{center}


Definindo entradas (saídas arredondadas para melhor visualização):

\begin{center}
    \begin{tabular}{|c|c|}
        \cline{1-2}
        lista & resultado esperado \\ \hline
        [] & None \\ \hline
        [('robo1',2,(4,4),5)] & [('robo1', 5.6, [(0, 0), (4, 4)])] \\ \hline
        [('robo1',2,(4,4),5),('robo3',7,(5,5),2)] & [('robo1', 5.6, [(0, 0), (4, 4)]), ('robo3', 7, [(0, 0), (5, 5)])] \\ \hline
    \end{tabular}
\end{center} 

\begin{center}
    \begin{tabular}{|c|}
        \cline{1-1}
        resultado obtido \\ \hline
        None \\ \hline
        [('robo1', 5.6, [(0, 0), (4, 4)])] \\ \hline
        [('robo1', 5.6, [(0, 0), (4, 4)]), ('robo3', 7, [(0, 0), (5, 5)])] \\ \hline
    \end{tabular}
\end{center}



\section{\texttt{merge\_ordenada\_tupla()}}
Dadas duas listas ordenadas pelo segundo elemento da tupla, junta-as de forma ordenada.

\begin{center}
    \begin{tabular}{|c|c|}
        \cline{1-2}
        lista & resultado esperado \\ \hline
        [] & [] \\ \hline
        [],[(a,b)] & [(a,b)] \\ \hline
        [(c,d)],[] & [(c,d)] \\ \hline
        [(a,b)],[(c,d)]  & [(a,b),(c,d)] \\ \hline
        [(a,b),(c,d)],[(b,c),(d,e)] & [(a,b),(b,c)(c,d),(d,e)] \\ \hline
    \end{tabular}
\end{center}

Definindo entradas:

\begin{center}
    \begin{tabular}{|c|c|c|}
        \cline{1-3}
        lista & resultado esperado & resultado obtido  \\ \hline
        [],[] & [] & []\\ \hline
        [(1,2)],[(3,4)] & [(1,2),(3,4)] & [(1,2),(3,4)] \\ \hline
        [(1,2),(3,4)],[(5,6),(7,8)] & [(1,2),(3,4),(5,6),(7,8)] & [(1,2),(3,4)],[(5,6),(7,8)] \\ \hline
    \end{tabular}
\end{center}


\section{\texttt{merge\_sort\_tupla()}}
Ordena uma lista de tuplas através do método \textit{merge sort}, tendo como referência o segundo elemento das tuplas.

\begin{center}
    \begin{tabular}{|c|c|}
        \cline{1-2}
        lista & resultado esperado \\ \hline
        [] & [] \\ \hline
        [(a)]  & [(a)] \\ \hline
        [(a,d),(c,b),(b,f)] & [(c,b),(a,d),(b,f)]  \\ \hline
    \end{tabular}
\end{center}


Definindo entradas:

\begin{center}
    \begin{tabular}{|c|c|c|}
        \cline{1-3}
        lista & resultado esperado & resultado obtido  \\ \hline
        [] & [] & [] \\ \hline
        [(1,2)] & [(1,2)] & [(1,2)] \\ \hline
        [(1,4),(2,3)] & [(2,3),(1,4)] & [(2,3),(1,4)] \\ \hline

    \end{tabular}
\end{center}


\section{\texttt{ids\_mais\_vitimas()}}
 A função retorna o(s) id(s) do(s) robô(s) que avistaram o maior número de vítimas.
 
 \begin{center}
    \begin{tabular}{|c|c|}
        \cline{1-2}
        lista & resultado esperado \\ \hline
        [] & None \\ \hline
        um robô  & o id do único robô \\ \hline
        mais de um robô, um robô com mais vítimas avistadas & id do robô com mais vítimas  \\ \hline
        mais de um robô, mais de um robô com mais vítimas & ids dos robôs com mais vítimas \\ \hline
    \end{tabular}
\end{center}

Definindo entradas:

\begin{center}
    \begin{tabular}{|c|c|c|}
        \cline{1-3}
        lista & resultado esperado & resultado obtido  \\ \hline
        [] & None & None \\ \hline
        [('robo1',1,(1,1),2)] & ['robo1'] & ['robo1'] \\ \hline
        [('robo1',1,(1,1),2),('robo2',5,(3,2),5)] & ['robo2'] & ['robo2'] \\ \hline
        [('robo1',1,(1,1),2),('robo2',5,(3,2),2)] & ['robo1', 'robo2'] & ['robo1', 'robo2'] \\ \hline
    \end{tabular}
\end{center} 

\end{apendicesenv}
