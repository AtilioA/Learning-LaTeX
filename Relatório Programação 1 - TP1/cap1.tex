\chapter*[INTRODUÇÃO]{INTRODUÇÃO}\label{cap-introducao}
\addcontentsline{toc}{chapter}{INTRODUÇÃO}

Este trabalho tem por objetivo documentar a estruturação de um projeto de auxílio à tomada de decisões dado um cenário hipotético de um ambiente acidentado. Neste, dados serão extraídos por robôs-célula que investigam o local, que serão processados de forma a aprimorar a atividade de resgate das vítimas que se encontram no ambiente. O plano de testes, presentes no Apêndice~\ref{apendiceA}, relata a confiabilidade das funções desenvolvidas.

No Capítulo~\ref{cap-planejamento-implementacao}, por sua vez, é abordada a forma como foram planejadas as resoluções de cada problema apresentado pelo contexto, assim como a implementação, de forma geral, das funções mais complexas. Por fim, a conclusão realiza uma breve avaliação da solução final, também pontuando comentários adicionais.

As referências para os métodos de testes, bem como paradigmas aplicativo e recursivo, que serão implementados podem ser encontrados nas notas de aula utilizadas pela disciplina, encontradas na página desta em junho de 2019. O projeto foi desenvolvido tendo como base a versão 3.6 ou 3.7 do Python.
