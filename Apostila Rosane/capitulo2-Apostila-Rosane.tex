\chapter{Lógica e Linguagem}

\section{Partículas Lógicas}

\subsection{Uso intuitivo dos juntores}

Estamos aqui interessados em estabelecer regras para o uso de certas partículas lógicas, como: \textbf{não}, \textbf{e},  \textbf{ou}, \textbf{se... então}, \textbf{se e somente se}, \textbf{nem} e \textbf{nor}, denominadas juntores ou conectivos lógicos (ou conjunções lógicas).
Sua função, como vimos na gramática anteriormente apresentada, é formar enunciados a partir de enunciados.
Para mostrar o uso técnico de tais partículas, com frequência buscamos exemplos da  linguagem corrente, aproveitando as convenções estabelecidas para esta linguagem.

É importante ressaltar, agora, que cada enunciado, no desenvolvimento de nosso trabalho, admitirá um, e um único valor lógico, isto é, será necessariamente verdadeiro ou  falso e nunca ambos.

\begin{enumerate}[label=(\arabic*)]
    \item \textbf{O juntor não} (simbolicamente: $\sim$)
\end{enumerate}
