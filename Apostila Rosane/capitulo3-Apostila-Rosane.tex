% Os módulos não têm preâmbulo e não precisam de \begin{document}, \end{document}. Basta escrever normalmente

\chapter{Cálculo Proposicional}

\section{Gramática do Cálculo Proposicional}

O Cálculo Proposicional (também chamado Cálculo dos Juntores) será, no presente capítulo, abordado de um ponto de vista semântico.
Antes, porém, cabe explicitar a gramática que permite definir suas fórmulas.
Uma vez fixados, pela intermediação da gramática, as expressões desse cálculo, resta mostrar os procedimentos que permitem interpretá-las.

No Cálculo Proposicional, o alfabeto consiste das duas seguintes classes de símbolos: i) \textbf{as letras proposicionais}; e ii) \textbf{os juntores}.
Conforme foi mencionado, usaremos as letras \textbf{p}, \textbf{q}, \textbf{r}, \textbf{s}, afetadas ou não de índices como letras proposicionais; e empregaremos os sinais $\sim, \land, \lor, \to, \iff$ como juntores.
Com este alfabeto e com as regras de formação do Cálculo Proposicional, torna-se possível caracterizarmos, de modo rigoroso, as fórmulas bem formadas do Cálculo Proposicional.

\subsection*{Regras de formação}

\begin{enumerate}[label={\roman*})]
    \item As letras proposicionais são fórmulas bem formadas (ditas fórmulas primas ou fórmulas atômicas).

    \item Se $\alpha$ e $\beta$ são fórmulas bem formadas, então ($\alpha \; \land \; \beta), (\alpha \; \lor \; \beta), (\alpha \to \beta), (\alpha \iff \beta), (\alpha \downarrow \beta), (\alpha \uparrow \beta), (\sim\alpha),$ são fórmulas bem formadas (ditas fórmulas compostas).

    \item As únicas fórmulas bem formadas do Cálculo Proposicional são as obtidas por um número finito de aplicações de i, ii.
\end{enumerate}

Daqui por diante, por mera questão de comodidade, usaremos apenas a palavra \textbf{fórmula}, em lugar da expressão \textbf{fórmula bem formada}.


\begin{exemplo}
    \bm{$((p \land q) \to (p \lor r))$} é uma fórmula bem formada do Cálculo Proposicional construída da seguinte forma:
\end{exemplo}

Como p, q, r  são fórmulas por i, segue-se daí e de ii que $(p \land q)$ e $(p \lor r)$ são fórmulas.
Logo, daí e de ii, tem-se que $((p \land q) \to (p \lor r))$ é fórmula bem formada.

\bigskip

\noindent \textbf{Comentário:} Por comodidade, omitiremos parênteses sempre que isto não causar ambiguidade. Assim, no caso acima, optaremos pela fórmula

% \begin{center}
%     $(p \land q) \to (p \lor r)$
% \end{center}

\centerline{$(p \land q) \to (p \lor r)$}

\section{Semântica}
No que vem a seguir, faremos um estudo intuitivo do que se costuma chamar a \textbf{Semântica do Cálculo Proposicional}.
Falaremos, então, em significado, interpretação, etc, comumente de modo um tanto vago.

O significado das fórmulas de um cálculo lógico é dado pela interpretação dessas fórmulas, isto é, pela atribuição apropriada de valores lógicos (V, F).
Assim, a semântica do Cálculo Proposicional consiste na interpretação de suas fórmulas.

Seja $\alpha$ uma fórmula. Uma interpretação de $\alpha$ consiste na atribuição de valores lógicos V  ou F às fórmulas atômicas componentes de $\alpha$, levando-se em consideração a interpretação de $\land,\ \lor,\ \to,\ \iff$, dadas pelas tabelas de verdade.

% Mudar isso depois
\begin{exemplo}
    Uma interpretação para a fórmula \bm{$p \to (q \lor r)$}
\end{exemplo}
\noindent consiste em atribuir valores lógicos a seus componentes atômicos p, q e r.
Isto se faz, esquematicamente, do seguinte modo:

% Falta centralizar os elementos da tabela
\begin{center}
    \begin{tabular}{l l l l}
        p & q & r & $p \to (q \lor r)$ \\ \hline
        V & F & F & F
    \end{tabular}
\end{center}

Esta fórmula é considerada então falsa segundo essa interpretação. Mas ela poderia ser verdadeiro segundo outra interpretação, por exemplo,

\begin{center}
    \begin{tabular}{l l l l}
        p & q & r & $p \to (q \lor r)$ \\ \hline
        F & V & F & V
    \end{tabular}
\end{center}

\begin{exemplo}
    Seja a fórmula \bm{$(p \lor q) \to (p \land q)$}
\end{exemplo}
\noindent que encerra dois componentes atômicos p e q; portanto, esta fórmula admite quatro interpretações, como indica a tabela a seguir:

\begin{center}
    % \large
    \begin{tabular}{l l l l l l}
                        & p & q & $p \lor q$ & $p \land q$ & $(p \lor q) \to (p \land q)$ \\ \hline
        interpretação 1 & V & V & V          & V            & V \\
        interpretação 2 & V & F & V          & F            & F \\
        interpretação 3 & F & V & V          & F            & F \\
        interpretação 4 & F & F & F          & F            & V
    \end{tabular}
\end{center}

De modo geral, se uma fórmula tiver n componentes atômicos distintos, esta fórmula admitirá $2^n$ interpretações.

\bigskip
\noindent \textbf{Convenção:} Para facilitar a leitura, vamos convencionar daqui em diante que uma interpretação será abreviada pela letra I afetada ou não de indices.


De um ponto de vista semântico, as noções de verdade, satisfatibilidade e validade são noções fundamentais.
A  seguir, passaremos ao  estudo dessas noções no  âmbito do  Cálculo Proposicional.

\bigskip
\textbf{Def.:} Admitamos que uma fórmula $\alpha$ tenha o valor V numa certa interpretação I. Neste caso, diz-se então que $\alpha$ é verdadeira nesta interpretação.

Assim, a fórmula $(p \lor q) \to (p \land q)$, do exemplo anterior, é verdadeira segundo a primeira interpretação.

\bigskip
\noindent \textbf{Comentário:} Note que se $\alpha$ e $\beta$ são fórmulas, então $\alpha \land \beta$ é \textbf{verdadeira} segundo uma interpretação se e somente se $\alpha$ e $\beta$ são ambas verdadeiras segundo esta interpretação.
