% Os módulos não têm preâmbulo e não precisam de \begin{document}, \end{document}. Basta escrever normalmente

\chapter{Cálculo Proposicional}

\section{Gramática do Cálculo Proposicional}

O Cálculo Proposicional (também chamado Cálculo dos Juntores) será, no presente capítulo, abordado de um ponto de vista semântico.
Antes, porém, cabe explicitar a gramática que permite definir suas fórmulas.
Uma vez fixados, pela intermediação da gramática, as expressões desse cálculo, resta mostrar os procedimentos que permitem interpretá-las.

No Cálculo Proposicional, o alfabeto consiste das duas seguintes classes de símbolos: i) \textbf{as letras proposicionais}; e ii) \textbf{os juntores}.
Conforme foi mencionado, usaremos as letras \textbf{p}, \textbf{q}, \textbf{r}, \textbf{s}, afetadas ou não de índices como letras proposicionais; e empregaremos os sinais $\sim, \wedge, \vee, \to, \iff$ como juntores.
Com este alfabeto e com as regras de formação do Cálculo Proposicional, torna-se possível caracterizarmos, de modo rigoroso, as fórmulas bem formadas do Cálculo Proposicional.

\subsection*{Regras de formação}

\noindent i) As letras proposicionais são fórmulas bem formadas (ditas fórmulas primas ou fórmulas atômicas).

\bigskip

\noindent ii) Se $\alpha$ e $\beta$ são fórmulas bem formadas, então ($\alpha \; \land \; \beta), (\alpha \; \vee \; \beta), (\alpha \to \beta), (\alpha \iff \beta), (\alpha \downarrow \beta), (\alpha \uparrow \beta), (\sim\alpha),$ são fórmulas bem formadas (ditas fórmulas compostas).
