\chapter{Cálculo Proposicional}

\section{Gramática do Cálculo Proposicional}

O Cálculo Proposicional (também chamado Cálculo dos Juntores) será, no presente capítulo, abordado de um ponto de vista semântico.
Antes, porém, cabe explicitar a gramática que permite definir suas fórmulas.
Uma vez fixados, pela intermediação da gramática, as expressões desse cálculo, resta mostrar os procedimentos que permitem interpretá-las.

No Cálculo Proposicional, o alfabeto consiste das duas seguintes classes de símbolos: i) \textbf{as letras proposicionais}; e ii) \textbf{os juntores}.
Conforme foi mencionado, usaremos as letras \textbf{p}, \textbf{q}, \textbf{r}, \textbf{s}, afetadas ou não de índices como letras proposicionais; e empregaremos os sinais $\sim, \land, \lor, \to, \iff$ como juntores.
Com este alfabeto e com as regras de formação do Cálculo Proposicional, torna-se possível caracterizarmos, de modo rigoroso, as fórmulas bem formadas do Cálculo Proposicional.

\subsection*{Regras de formação}

\begin{enumerate}[label={\roman*})]
    \item As letras proposicionais são fórmulas bem formadas (ditas fórmulas primas ou fórmulas atômicas).

    \item Se $\alpha$ e $\beta$ são fórmulas bem formadas, então ($\alpha \; \land \; \beta), (\alpha \; \lor \; \beta), (\alpha \to \beta), (\alpha \iff \beta), (\alpha \downarrow \beta), (\alpha \uparrow \beta), (\sim\alpha),$ são fórmulas bem formadas (ditas fórmulas compostas).

    \item As únicas fórmulas bem formadas do Cálculo Proposicional são as obtidas por um número finito de aplicações de i, ii.
\end{enumerate}

Daqui por diante, por mera questão de comodidade, usaremos apenas a palavra \textbf{fórmula}, em lugar da expressão \textbf{fórmula bem formada}.


\begin{exemplo}
    \bm{$((p \land q) \to (p \lor r))$} é uma fórmula bem formada do Cálculo Proposicional construída da seguinte forma:
\end{exemplo}

Como p, q, r  são fórmulas por i, segue-se daí e de ii que $(p \land q)$ e $(p \lor r)$ são fórmulas.
Logo, daí e de ii, tem-se que $((p \land q) \to (p \lor r))$ é fórmula bem formada.

\bigskip

\noindent \textbf{Comentário:} Por comodidade, omitiremos parênteses sempre que isto não causar ambiguidade. Assim, no caso acima, optaremos pela fórmula

% \begin{center}
%     $(p \land q) \to (p \lor r)$
% \end{center}

\centerline{$(p \land q) \to (p \lor r)$}

\section{Semântica}
\setcounter{exemplo}{0}
No que vem a seguir, faremos um estudo intuitivo do que se costuma chamar a \textbf{Semântica do Cálculo Proposicional}.
Falaremos, então, em significado, interpretação, etc, comumente de modo um tanto vago.

O significado das fórmulas de um cálculo lógico é dado pela interpretação dessas fórmulas, isto é, pela atribuição apropriada de valores lógicos (V, F).
Assim, a semântica do Cálculo Proposicional consiste na interpretação de suas fórmulas.

Seja $\alpha$ uma fórmula. Uma interpretação de $\alpha$ consiste na atribuição de valores lógicos V  ou F às fórmulas atômicas componentes de $\alpha$, levando-se em consideração a interpretação de $\land,\ \lor,\ \to,\ \iff$, dadas pelas tabelas de verdade.

% Mudar isso depois
\begin{exemplo}
    Uma interpretação para a fórmula \bm{$p \to (q \lor r)$}
\end{exemplo}
\noindent consiste em atribuir valores lógicos a seus componentes atômicos p, q e r.
Isto se faz, esquematicamente, do seguinte modo:

\begin{center}
    \begin{tabular}{c c c c}
        p & q & r & $p \to (q \lor r)$ \\ \hline
        V & F & F & F
    \end{tabular}
\end{center}

Esta fórmula é considerada então falsa segundo essa interpretação. Mas ela poderia ser verdadeiro segundo outra interpretação, por exemplo,

\begin{center}
    \begin{tabular}{c c c c}
        p & q & r & $p \to (q \lor r)$ \\ \hline
        F & V & F & V
    \end{tabular}
\end{center}

\begin{exemplo}
    Seja a fórmula \bm{$(p \lor q) \to (p \land q)$}
\end{exemplo}
\noindent que encerra dois componentes atômicos p e q; portanto, esta fórmula admite quatro interpretações, como indica a tabela a seguir:

\begin{center}
    % \large
    \begin{tabular}{c c c c c c}
                        & p & q & $p \lor q$ & $p \land q$ & $(p \lor q) \to (p \land q)$ \\ \hline
        interpretação 1 & V & V & V          & V            & V \\
        interpretação 2 & V & F & V          & F            & F \\
        interpretação 3 & F & V & V          & F            & F \\
        interpretação 4 & F & F & F          & F            & V
    \end{tabular}
\end{center}

De modo geral, se uma fórmula tiver n componentes atômicos distintos, esta fórmula admitirá $2^n$ interpretações.

\bigskip
\noindent
\textbf{Convenção:} Para facilitar a leitura, vamos convencionar daqui em diante que uma interpretação será abreviada pela letra I afetada ou não de indices.

De um ponto de vista semântico, as noções de verdade, satisfatibilidade e validade são noções fundamentais.
A  seguir, passaremos ao  estudo dessas noções no  âmbito do  Cálculo Proposicional.

\bigskip
\noindent
\textbf{Def.:} Admitamos que uma fórmula $\alpha$ tenha o valor V numa certa interpretação I.
Neste caso, diz-se então que $\alpha$ é verdadeira nesta interpretação.

Assim, a fórmula $(p \lor q) \to (p \land q)$, do exemplo anterior, é verdadeira segundo a primeira interpretação.

\bigskip
\noindent
\textbf{Comentário:} Note que se $\alpha$ e $\beta$ são fórmulas, então $\alpha \land \beta$ é \textbf{verdadeira} segundo uma interpretação se e somente se $\alpha$ e $\beta$ são ambas verdadeiras segundo esta interpretação.

\bigskip
\noindent
\textbf{Def.:} Admitamos que uma fórmula $\alpha$ seja verdadeira segundo alguma interpretação.
Neste caso, diz-se que $\alpha$ é \textbf{satisfatível} ou \textbf{consistente}.

A fórmula $p \to (q \lor r)$ do exemplo anterior é satisfatível.


\bigskip
\noindent
\textbf{Def.:} Uma fórmula $\alpha$ é \textbf{válida} quando é verdadeira em todas as suas interpretações.

\bigskip
\noindent
\textbf{Comentário:} As formulas válidas do  Cálculo Proposicional são também chamadas de \textbf{tautologias}.

Note-se que no Cálculo Proposicional uma fórmula é \textbf{válida} ou \textbf{tautológica} se for verdadeira em todas as possíveis atribuições de valores lógicos a suas fórmulas atômicas.
Além disso, se $\alpha$ e $\beta$ são fórmulas, então $\alpha \land \beta$ é válida somente se e $\alpha$ e $\beta$ são ambas válidas.

\begin{exemplo}
    Seja a fórmula $(p \land (q \to p)) \to p$
\end{exemplo}
\noindent cuja tabela verdade é

\begin{center}
    % \large
    \begin{tabular}{c c c c c c}
        p & q & $q \to p$ & $p \land (q \to p)$ & $(p \land (q \to p)) \to p$ \\ \hline
        V & V & V         & V                   & \textbf{V} \\
        V & F & V         & V                   & \textbf{V} \\
        F & V & F         & F                   & \textbf{V} \\
        F & F & V         & F                   & \textbf{V}
    \end{tabular}
\end{center}
Vemos que a fórmula acima é válida \textbf{--} sendo portanto uma tautologia \textbf{--} já que é verdadeira em todas as interpretações.

\bigskip
\noindent
\textbf{Def.:} Admitamos que uma fórmula $\alpha$ tenha valor \textbf{F} numa interpretação I.
Neste caso, diz-se então que $\alpha$ é falsa segundo I.

\noindent Assim, a fórmula

\centerline{$(p \lor q) \to (p \land q)$}

\noindent é falsa de acordo com a segunda interpretação (vide Exemplo 1).

\bigskip
\noindent
\textbf{Comentário:} Note-se que se $\alpha$ e $\beta$ são fórmulas, então $\alpha \lor \beta$ é \textbf{falsa} segundo uma interpretação se e somente se $\alpha$ e $\beta$ são ambas falsas segundo esta interpretação.


\bigskip
\noindent
\textbf{Def.:} Uma fórmula $\alpha$ é \textbf{insatisfatível} ou \textbf{inconsistente}  quando for falsa segundo qualquer interpretação.

\bigskip
\noindent
\textbf{Comentário:} As formulas insatisfatíveis do  Cálculo Proposicional são também chamadas de \textbf{contradições}.
Note que uma contradição é a negação de uma tautologia.

\begin{exemplo}
    Seja a fórmula $p\ \land \sim p$
\end{exemplo}
\noindent cuja tabela verdade é

\begin{center}
    \begin{tabular}{c c c}
        p & q & $p\ \land ~p$ \\ \hline
        V & F & F \\
        F & V & F
    \end{tabular}
\end{center}
é insatisfatível, pois é falsa segundo todas as interpretações.

\bigskip
\noindent
\textbf{Comentário:} Note-se que se $\alpha$ e $\beta$ são fórmulas, então $\alpha\ \lor\ \beta$ é \textbf{insatisfatível} se e somente se $\alpha$ e $\beta$ são simultaneamente insatisfatíveis.

\bigskip
\noindent
\textbf{Def.:} Uma fórmula $\alpha$ será \textbf{inválida} quando for falsa segundo alguma interpretação.

\begin{exemplo}
    Seja a fórmula $p \to q$
\end{exemplo}
\noindent cuja tabela verdade é

\begin{center}
    \begin{tabular}{c c c}
        p & q & $p\ \to ~q$ \\ \hline
        V & V & V \\
        V & F & F \\
        F & V & V \\
        F & F & V
    \end{tabular}
\end{center}

\noindent é inválida, pois é falsa com respeito à segunda interpretação.

\bigskip
Das definições apresentadas, chegamos às seguintes conclusões:

\begin{enumerate}[label=\textbf{{\arabic*}.}]
    \item Uma fórmula é \textbf{válida} se e somente se sua negação for \textbf{insatisfatível}.
    \item Uma fórmula é \textbf{insatisfatível} ou \textbf{inconsistente} se e somente se sua negação for \textbf{válida}.
    \item Uma fórmula é \textbf{inválida} se e somente se existir pelo menos uma \textbf{interpretação} em que ela é falsa.
    \item Uma fórmula é \textbf{satisfatível} ou \textbf{consistente} se e somente se existir pelo menos uma \textbf{interpretação} segundo a qual ela é verdadeira.
    \item Se uma fórmula for \textbf{válida}, então é \textbf{satisfatível}.
    \item Se uma fórmula for \textbf{insatisfatível}, então é \textbf{inválida}.
\end{enumerate}

No Cálculo Proposicional, as fórmulas que não são tautologias e nem contradições são comumente chamadas de \textbf{contingentes}.

A título de ilustração, sejam os seguintes exemplos:

\begin{center}
    \begin{tabular}{c c c}
        \textbf{TAUTOLÓGICAS} & \textbf{CONTINGENTES} & \textbf{CONTRADIÇÕES}\\
        $p\ \lor \sim p$ & $p$ & $\sim(p \to p \lor q)$\\
        $p \to p$ & $p \land q \to r$ & $\sim(p\ \lor \sim p)$\\
        $p \to\ p \lor q$ & $p \to q$ & $p\ \land \sim p$\\
        $p \land q \to q$ & $p \to\ \sim q$ & $\sim(p \to p)$
    \end{tabular}
\end{center}

\bigskip
\noindent
\textbf{Def.:} Sejam $\beta_1,\ \beta_2,\ \alpha$ fórmulas. Dizemos que $\alpha$ é \textbf{consequência lógica}  de $\beta_1,\ \beta_2$, quando cada interpretação I que torna $\beta_1,\ \beta_2$ verdadeira, torna $\alpha$ verdadeira.

\bigskip
\noindent
\textbf{Comentário:} Se $\alpha$ é consequência lógica de $\beta_1,\ \beta_2$, então diz-se também que $\alpha$ segue-se logicamente de $\beta_1,\ \beta_2$. Simbolicamente, indica-se que $\alpha$ é consequência lógica de $\beta_1,\ \beta_2$ mediante a seguinte notação:

\centerline{$\beta_1,\ \beta_2\ \vDash \alpha$} % \models

\begin{exemplo}
    Vemos que $\sim p$ é consequência lógica de $p \to q$ e de $p \to\ \sim q$, pois, através da tabela verdade, temos
\end{exemplo}

\begin{center}
    \begin{tabular}{c c c c c c c}
              & p & q & $\sim p$ & $\sim q$ & $p \to q$ & $p \to\ \sim q$ \\ \hline
        $I_1$ & V & V & F        & F        & V         & F \\
        $I_2$ & V & F & F        & V        & F         & V \\
        $I_3$ & F & V & V        & F        & V         & V \\
        $I_4$ & F & F & V        & V        & V         & V \\
    \end{tabular}
\end{center}
que para as interpretações $I_3$ e $I_4$ em que \bm{$p \to q$} e \bm{$p \to \sim q$} são verdadeiras, $\sim p$ é também verdadeira.

\bigskip
\noindent
\textbf{Def.:} Sejam $\beta_1,\ \dots, \beta_n,\ \alpha$ fórmulas. Dizemos que $\alpha$ é consequência lógica de $\beta_1,\dots, \beta_n$ quando cada interpretação de I que torna cada $\beta_j\ (1 \leq j \leq n)$ simultaneamente verdadeira torna $\alpha$ verdadeira.

\bigskip
\noindent
\textbf{Comentário:} Se $\alpha$ é consequência lógica de $\beta_1,\dots, \beta_n$, então diz-se também que $\alpha$ segue-se logicamente de $\beta_1,\dots, \beta_n$. Simbolicamente, indica-se que $\alpha$ é consequência lógica de $\beta_1,\dots, \beta_n$ mediante a seguinte notação:

\centerline{$\beta_1,\dots, \beta_n \vDash \alpha$}

\bigskip
\noindent
\textbf{Comentário:} Simbolicamente, indica-se que $\alpha$ é uma fórmula válida mediante a seguinte notação:

\centerline{$\vDash \alpha$}

Da definição de consequência lógica, obtemos o enunciado a seguir, que pode ser visto como definição alternativa para consequência lógica:

\begin{center}
    $\alpha$ é consequência lógica de $\beta_1,\dots, \beta_n$ se e somente se $\beta_1, \land \dots, \land\ \beta_n \to \alpha$ é uma tautologia.
\end{center}

\textbf{Simbolicamente}:

$\beta_1,\dots, \beta_n \vDash \alpha\ \iff\ \beta_1, \land \dots, \land\ \beta_n\footnote{Como na Lógica Proposicional o juntor $\land$ é associativo, usaremos $\alpha_1 \land \alpha_2 \land \alpha_3$ em vez de $(\alpha_1 \land \alpha_2) \land \alpha_3$.} \to \alpha$ é uma tautologia.

\bigskip
\noindent Assim, $\sim p$ é consequência lógica de

\centerline{\bm{$p \to q$} e \bm{$p \to \sim q$}}
\noindent pois

\centerline{($p \to q) \land (p \to\ \sim q) \to\ \sim q$ é uma tautologia}

\newpage
Tal fato pode ser constatado mediante a tabela de verdade que se segue:
\begin{center}
    \begin{tabular}{c c c c c c c c}
        p & q & $\sim p$ & $\sim q$ & $p \to q$ & $p \to\ \sim q$ & $(p \to q) \land (p \to\ \sim q)$ \\ \hline
        V & V & F        & F        & V         & F               & F \\
        F & V & V        & F        & V         & V               & V \\
        V & F & F        & V        & F         & V               & V \\
        F & F & V        & V        & V         & V               & V \\
    \end{tabular}
\end{center}

\begin{center}
    \begin{tabular}{c}
        $(p \to q) \land (p \to\ \sim q) \to\ \sim p$ \\ \hline
        V \\
        V \\
        V \\
        V \\
    \end{tabular}
\end{center}

\bigskip
\noindent
\textbf{Def.:}  Dizemos que uma fórmula $\alpha$ é \textbf{logicamente} (ou semanticamente) \textbf{equivalente} a uma fórmula $\beta$ quando $\alpha$ é \textbf{consequência lógica} de $\beta$ e $\beta$ é consequência lógica de $\alpha$.

\bigskip
\noindent
\textbf{Obs.:}  Uma fórmula $\alpha$ é \textbf{logicamente equivalente} a uma fórmula $\beta$ se e somente se a fórmula $\alpha \iff \beta$ for \textbf{válida} (i.e., $\vDash \alpha \iff \beta$).  Por comodidade, escrevemos

\centerline{$\alpha \equiv \beta$}
\noindent para indicar que $\alpha$ é logicamente (ou semanticamente) equivalente a $\beta$; logo,

$\alpha \equiv \beta$ se e somente se $\alpha \iff \beta$ é uma tautologia.

\bigskip
A seguir, apresentamos algumas equivalências lógicas importantes. Notemos que todas estas fórmulas são tautologias.

\begin{enumerate}[label={\arabic*}.]
    \item $\alpha \equiv\ \sim \sim \alpha$
    \item $\alpha \land \beta \equiv \beta \land \alpha$
    \item $\alpha \lor \beta \equiv \beta \lor \alpha$
    \item $(\alpha \iff \beta) \equiv (\alpha \to \beta\ \land\ \beta \to \alpha)$
    \item $\alpha \to \beta \equiv\ \sim \alpha \lor \beta$
    \item $\sim (\alpha \land \beta) \equiv\ \sim \alpha\ \lor \sim \beta$
    \item $\sim (\alpha \lor \beta) \equiv\ \sim \alpha\ \land \sim \beta$
    \item $\alpha \lor (\beta \land \delta) \equiv (\alpha \lor \beta) \land (\alpha \lor \delta)$
    \item $\alpha \land (\beta \lor \delta) \equiv (\alpha \land \beta) \lor (\alpha \land \delta)$
    \item $(\alpha \to \beta \land \delta) \equiv (\alpha \to \beta) \land (\alpha \to \delta)$
    \item $(\alpha \lor \beta \to \delta) \equiv (\alpha \to \delta) \land (\beta \to \delta)$
    \item $\alpha \to (\beta \to \delta) \equiv (\alpha \land \beta) \to \delta$
    \item $(\alpha \to \beta) \to \delta \equiv (\alpha \lor \delta) \lor (\beta \to \delta)$
    \item $(\alpha\ \land \sim \beta) \to \delta \equiv \alpha \to (\beta \lor \delta)$
\end{enumerate}

\newpage

\newpage

\section{Formas Normais}
Para o estudo de formas clausais, conceito básico para as linguagens lógicas de programação, é importante a noção de forma normal. Aqui, desenvolveremos, de modo informal, os conceitos de forma conjuntiva normal e forma disjuntiva normal.

\bigskip
\noindent
\textbf{Def.:}  Literais são formas atômicas ou negações de fórmulas atômicas.

\begin{defi}
Literais são formas atômicas ou negações de fórmulas atômicas.
\end{defi}

\begin{itemize}[label={},itemindent=-3em,leftmargin=\parindent]
    \item \textbf{Def.:} Literais são formas atômicas ou negações de fórmulas atômicas.
\end{itemize}

\subsection{Forma Normal Conjuntiva (FNC)}
