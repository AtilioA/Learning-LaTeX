\documentclass[
	% -- opções da classe memoir --
	14pt,				% tamanho da fonte
	% openright,			% capítulos começam em página ímpar (insere página vazia caso preciso)
	twoside,			% para impressão em recto e verso. Oposto a oneside
	a4paper,			% tamanho do papel.
	% -- opções da classe abntex2 --
	%chapter=TITLE,		% títulos de capítulos convertidos em letras maiúsculas
	%section=TITLE,		% títulos de seções convertidos em letras maiúsculas
	%subsection=TITLE,	% títulos de subseções convertidos em letras maiúsculas
	%subsubsection=TITLE,% títulos de sub-subseções convertidos em letras maiúsculas
	% -- opções do pacote babel --
	english,			% idioma adicional para hifenização
	french,				% idioma adicional para hifenização
	spanish,			% idioma adicional para hifenização
	brazil,				% o último idioma é o principal do documento
    ]{abntex2}


\usepackage{lmodern}			% Usa a fonte Latin Modern
\usepackage[T1]{fontenc}		% Seleção de códigos de fonte.
\usepackage[utf8]{inputenc}		% Codificação do documento (conversão automática dos acentos)
\usepackage{indentfirst}		% Indenta o primeiro parágrafo de cada seção.
\usepackage{color}				% Controle das cores
\usepackage{graphicx}			% Inclusão de gráficos
\usepackage{microtype} 			% para melhorias de justificação

% Em geral, pacotes para tornar a aparência mais próxima da original da apostila
\usepackage{float}
\usepackage{mathabx}
%\usepackage{amsmath}
\usepackage[LGRgreek]{mathastext}
\usepackage{multicol}
\usepackage{multirow}
\usepackage{lipsum}
\usepackage[ampersand]{easylist}
\usepackage{enumitem}


% Cria comando para repetir um caractere n vezes
\newcommand\myrepeat[2]{%
  \begingroup
  \lccode`m=`#2\relax
  \lowercase\expandafter{\romannumeral#1000}%
  \endgroup
}

\newenvironment{tightcenter}{%
  \setlength\topsep{0pt}
  \setlength\parskip{0pt}
  \begin{center}
}{%
  \end{center}
}
\renewcommand{\baselinestretch}{1.5}

% Tamanho do recuo do parágrafo
\setlength\parindent{2cm}


% Fim do preâmbulo
\begin{document}

% Estilo do documento
\makepagestyle{memoir}
\makeevenfoot{memoir}{}{}{\thepage}
\makeoddfoot{memoir}{}{}{\thepage}
\pagestyle{plain}

\chapter{Linguagem}
\section{Linguagem Natural}

\thispagestyle{plain}


Linguagem natural é qualquer linguagem que os seres humanos aprendem em seu
ambiente de vida e usam para a comunicação com outros seres humanos, como é o
caso do português.

Embora seja interessante o estudo de todas as formas de linguagem e dos diversos
fenômenos linguísticos, estes não serão aqui aprofundados. Faremos apenas alguns
comentários para indicar a necessidade de construirmos uma linguagem artificial
para o estudo da Lógica.

\newpage

\lipsum[50]

\lipsum[50]

\lipsum[50]

\lipsum[50]

\newpage

\section{Linguagem Simbólica}
Por ser vaga e ambígua, a linguagem corrente ou natural não é inteiramente adequada para veicular os resultados oriundos da investigação científica, sobretudo no domínio das ciências formais (Matemática, Lógica, Computação, etc.).
Por este motivo, para fixar os princípios, métodos e resultados de nossas investigações no plano da Lógica e da Computação, vamos aqui nos servir não de uma linguagem natural, mas de uma linguagem formal ou simbólica ou artificial.
A introdução de linguagens artificiais com o objetivo de exprimir com correção e exatidão o pensamento e os resultados do conhecimento científico é apenas uma de suas vantagens.
Outra, não menos importante, é a função de tornar sintético ou conciso o pensamento.
Com efeito, a linguagem simbólica objetiva também torna concisas construções que, em linguagem corrente, seriam extremamente longas.
O exemplo abaixo mostra com toda clareza o que queremos dizer:

\vskip 1cm

\begin{center}
    \noindent\texttt{\myrepeat{76}{-}}
\end{center}
\vspace{-1\baselineskip} % Diminui espaço entre linhas

\begingroup
% \leftskip=3cm \rightskip=3cm
% \parfillskip=0.5cm plus 1fil

\leftskip=3.3cm plus .fil \rightskip=3.3cm
\parfillskip=0.5cm plus 1fil
\noindent O produto de um número pela soma de dois outros é igual ao produto do primeiro pelo segundo somado ao produto do primeiro pelo terceiro.

\vspace{-1\baselineskip} % Diminui espaço entre linhas
\begin{center}
    \noindent\texttt{\myrepeat{76}{-}}
\end{center}

% \leftskip=3.3cm plus .fil \rightskip=3.3cm
% \parfillskip=0.5cm plus 1fil

\centering simbolicamente

\begin{center}
    \noindent\texttt{\myrepeat{80}{-}}
\end{center}
\vspace{-1\baselineskip} % Diminui espaço entre linhas

\leftskip=3.6cm plus .fil \rightskip=3cm
\parfillskip=0.5cm plus 1fil
\noindent \centering se $x$, $y$, $z$ são números arbitrários,\\
$x \cdot (y + z) = x \cdot y + x \cdot z$

\vspace{-1\baselineskip} % Diminui espaço entre linhas

\begin{center}
    \noindent\texttt{\myrepeat{80}{-}}
\end{center}


\endgroup

\newpage

Este exemplo estabelece de modo inequívoco os dois seguintes resultados. Em primeiro lugar, o enunciado em linguagem portuguesa é menos claro e exato que aquele que foi expresso em linguagem simbólica.
Em segundo lugar, um é muito mais conciso do que o outro.
Mais tarde, veremos como precisar ainda mais a tradução acima pela introdução de símbolos lógicos.

\newpage

\noindent São exemplos de termos:

\begin{itemize}[itemsep=0.01pt]
\renewcommand\labelitemi{\textbf{-}}
    \item $x$
    \item recursividade
    \item PROLOG
    \item  O autor de Mar Morto
    \item  A linguagem de programação baseada na linguagem ALGOL
    \item $A \cap B$
    \item Maria
    \item 3
    \item (1, 2)
    \item $5 \times 7$
    \item $x + y$
    \item $y!$
    \item \dots
    \item caramba

\end{itemize}

\textbf{Enunciado} (ou proposição) é a expressão que correlaciona objetos ou descreve propriedades de objetos.

\noindent São exemplos de termos:

\begin{itemize}[itemsep=0.01pt]
\renewcommand\labelitemi{\textbf{-}}
    \item PASCAL é uma linguagem de programação de alto nível
    \item $x + y = 3$
    \item  = (3, + (x,y))
    \item  maior(3, 2)
    \item professor(Ivo, Matemática)
    \item ensina(Ivo, João)
    \item Piquet é campeão do mundo
    \item Todo nº par é divisível por 2
    \item ou Pedro estuda ou Pedro será reprovado no nivelamento
    \item $x^2 - 5x + 6 = 0$
    \item pertence(México, América)
\end{itemize}

\newpage



\end{document}
