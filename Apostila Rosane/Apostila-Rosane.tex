\documentclass[
	% -- opções da classe memoir --
	14pt,				% Tamanho da fonte
	% openright,			% Capítulos começam em página ímpar (insere página vazia caso preciso)
	twoside,			% Para impressão em recto e verso. Oposto a oneside
	a4paper,			% Tamanho do papel.
	% -- opções da classe abntex2 --
	%chapter=TITLE,		% Títulos de capítulos convertidos em letras maiúsculas
	%section=TITLE,		% Títulos de seções convertidos em letras maiúsculas
	%subsection=TITLE,	% Títulos de subseções convertidos em letras maiúsculas
	%subsubsection=TITLE,% Títulos de sub-subseções convertidos em letras maiúsculas
	% -- opções do pacote babel --
	english,			% Idioma adicional para hifenização
	french,				% Idioma adicional para hifenização
	spanish,			% Idioma adicional para hifenização
	brazil,				% O último idioma é o principal do documento
    ]{abntex2}


\usepackage{lmodern}			% Usa a fonte Latin Modern
\usepackage[T1]{fontenc}		% Seleção de códigos de fonte.
\usepackage[utf8]{inputenc}		% Codificação do documento (conversão automática dos acentos)
\usepackage{indentfirst}		% Indenta o primeiro parágrafo de cada seção.
\usepackage{color}				% Controle das cores
\usepackage{graphicx}			% Inclusão de gráficos
\usepackage{microtype} 			% Para melhorias de justificação

% Em geral, pacotes para tornar a aparência mais próxima da original da apostila
\usepackage{float}
\usepackage{lipsum}
\usepackage{amsmath}
% Possui comandos interessantes e.g. '\implies' mas que são substituídos sem perda alguma
% \usepackage{mathabx} % Conflita com amsmath
\usepackage[LGRgreek]{mathastext}
\usepackage{multicol}
\usepackage{multirow}
\usepackage{lipsum}
\usepackage[ampersand]{easylist}
\usepackage{enumitem}
\usepackage{scrextend}
\usepackage{bm}
\usepackage{array}
\usepackage{amssymb}

% Cria comando para repetir um caractere n vezes
\newcommand\myrepeat[2]{%
  \begingroup
  \lccode`m=`#2\relax
  \lowercase\expandafter{\romannumeral#1000}%
  \endgroup
}

\newenvironment{tightcenter}{%
  \setlength\topsep{0pt}
  \setlength\parskip{0pt}
  \begin{center}
}{%
  \end{center}
}
\renewcommand{\baselinestretch}{1.5}

% Ambiente para recuar texto (não está funcionando)
\newenvironment{textorecuado}
{
    \hskip 3cm
}


% Tamanho do recuo do parágrafo
\setlength\parindent{2cm}

% Tamanho da separação das colunas e linhas em tabelas (array)
\setlength{\tabcolsep}{12pt}
\setlength\extrarowheight{12pt}


% Fim do preâmbulo
\begin{document}

% Estilo do documento
\makepagestyle{memoir}
\makeevenfoot{memoir}{}{}{\thepage}
\makeoddfoot{memoir}{}{}{\thepage}
\pagestyle{plain}

\chapter{Linguagem}
\section{Linguagem natural}

\thispagestyle{plain}


Linguagem natural é qualquer linguagem que os seres humanos aprendem em seu
ambiente de vida e usam para a comunicação com outros seres humanos, como é o
caso do português.

Embora seja interessante o estudo de todas as formas de linguagem e dos diversos
fenômenos linguísticos, estes não serão aqui aprofundados. Faremos apenas alguns
comentários para indicar a necessidade de construirmos uma linguagem artificial
para o estudo da Lógica.

Dada a íntima relação entre raciocínio e linguagem, impõe-se à Lógica o exame da linguagem.
Mas cabe não nos esquecermos de que a linguagem natural é vaga, ambígua e imprecisa, sendo adequada para a poesia, literatura e o folclore, mas não para a ciência e a tecnologia.

Embora haja autores que se sirvam da linguagem natural no estudo da Lógica --- por força das dificuldades de traduzir-se sentenças da linguagem corrente para uma linguagem artificial --- aqui não vamos seguir esta abordagem.

Tentaremos contornar os obstáculos, cientes de que, linguagens simbólicas, até hoje elaboradas, não são instrumentos adequados para o tratamento de todas as formas que podem assumir o raciocínio humano.

\newpage

Chamamos de \emph{linguagem artificial} % remover
toda linguagem deliberadamente construída para fins específicos, como as \textbf{linguagens de programação} e os diversos cálculos da Lógica e da Matemática.
Não vamos aqui classificar as diversas linguagens, mas apenas apresentar linguagens artificias que sejam aptas ao estudo da Lógica e da Informática.

A linguagem que surgiu na Grécia e evoluiu cada vez mais simbolicamente até o século XIX, considerada instrumento imprescindível para o tratamento da Matemática, não era, na verdade, suficientemente precisa para o tratamento das complexas questões que aparecem com a Matemática contemporânea.
Mas ela se aprimorou, de início, com os trabalhos de Frege sobre sistemas formais.
E, posteriormente, com o surgimento de paradoxos no âmbito da teoria dos conjuntos, houve a necessidade de uma reanálise e de uma retificação do aparato formal dos sistemas dedutivos convencionais.
Tornou-se, assim, necessário introduzir uma linguagem precisa, definida por uma gramática explícita, gramática esta cujas regras não possuem exceções. Esta linguagem, chamada \emph{linguagem formal}, % remover
será o objeto de estudo deste capítulo.
Só estas linguagens são aptas para o tratamento da Matemática e da Informática;
só com o emprego destes tipos de linguagens é possível construir e operar com os atuais computadores.

\newpage

\section{Linguagem simbólica}
Por ser vaga e ambígua, a linguagem corrente ou natural não é inteiramente adequada para veicular os resultados oriundos da investigação científica, sobretudo no domínio das ciências formais (Matemática, Lógica, Computação, etc.).
Por este motivo, para fixar os princípios, métodos e resultados de nossas investigações no plano da Lógica e da Computação, vamos aqui nos servir não de uma linguagem natural, mas de uma linguagem formal ou simbólica ou artificial.
A introdução de linguagens artificiais com o objetivo de exprimir com correção e exatidão o pensamento e os resultados do conhecimento científico é apenas uma de suas vantagens.
Outra, não menos importante, é a função de tornar sintético ou conciso o pensamento.
Com efeito, a linguagem simbólica objetiva também torna concisas construções que, em linguagem corrente, seriam extremamente longas.
O exemplo abaixo mostra com toda clareza o que queremos dizer:

\vskip 1cm

\begin{center}
    \noindent\texttt{\myrepeat{76}{-}}
\end{center}
\vspace{-1\baselineskip} % Diminui espaço entre linhas

\begingroup

    \leftskip=3.3cm plus .fil \rightskip=3.3cm
    \parfillskip=0.5cm plus 1fil
    \noindent O produto de um número pela soma de dois outros é igual ao produto do primeiro pelo segundo somado ao produto do primeiro pelo terceiro.

    \vspace{-1\baselineskip} % Diminui espaço entre linhas
    \begin{center}
        \noindent\texttt{\myrepeat{76}{-}}
    \end{center}


    \centering simbolicamente

    \begin{center}
        \noindent\texttt{\myrepeat{80}{-}}
    \end{center}
    \vspace{-1\baselineskip} % Diminui espaço entre linhas

    \leftskip=3.6cm plus .fil \rightskip=3cm
    \parfillskip=0.5cm plus 1fil
    \noindent \centering se $x$, $y$, $z$ são números arbitrários,\\
    $x \cdot (y + z) = x \cdot y + x \cdot z$

    \vspace{-1\baselineskip} % Diminui espaço entre linhas

    \begin{center}
        \noindent\texttt{\myrepeat{80}{-}}
    \end{center}

\endgroup

\newpage

Este exemplo estabelece de modo inequívoco os dois seguintes resultados.
Em primeiro lugar, o enunciado em linguagem portuguesa é menos claro e exato que aquele que foi expresso em linguagem simbólica.
Em segundo lugar, um é muito mais conciso do que o outro.
Mais tarde, veremos como precisar ainda mais a tradução acima pela introdução de símbolos lógicos.

No desenvolvimento desse trabalho, mostraremos as vantagens da simbolização ao evidenciar a facilidade com que se realiza a análise de um termo, de uma sentença ou de uma demonstração, quando estes forem escritos em adequada notação simbólica.
Veremos ainda que muitas vezes essa linguagem nos permite até deixar de considerar questões de conteúdo envolvidas pelos termos, sentenças e demonstrações.

Para iniciar nossas considerações, admitamos um \textbf{alfabeto} formado por todos os símbolos matemáticos e letras dos alfabetos latino e grego.
Por exemplo: a, $\alpha$, $+$, $\epsilon$, e outros.

A partir dos símbolos do alfabeto e da noção intuitiva de concatenação horizontal, podemos formar as expressões mais arbitrárias possíveis.

São exemplos de expressões:

\vskip 1cm

% Criar ambiente para \hskip
\hskip 2.5cm $a + y$, $3 \in (3, 5, 7), +-xy$,\\
\hskip 2.5cm Linguagem de Programação Fortran,\\
\hskip 2.5cm $xy-$, $+xy$, e outros.

\newpage

Mas esta regra formadora irrestrita leva-nos inevitavelmente a expressões que não nos interessam.
Por exemplo, \textit{abce} não é, até o momento, uma palavra da língua portuguesa; do mesmo modo, $+\,\, 3 \,=\,\,\, \in\, 7\,\, x$ nem se refere a um objeto, nem é uma sentença ou expressão da Matemática.
Consideremos agora duas classes de expressões bem formadas que cabem ser devidamente destacadas.

Por tentativa e erro, procuraremos distinguir, entre as expressões bem formadas, aquelas chamadas \textbf{TERMO} daquelas chamadas \textbf{ENUNCIADO}.

\vskip 0.5cm

\begin{addmargin}[2.0cm]{0cm}
\textbf{Termo} é a expressão que nomeia ou descreve um objeto (termo fechado ou nome) ou é a expressão que resulta em um nome ou descrição de um objeto quando as variáveis que nela ocorrem são substituídas por nomes ou descrições de objetos (termo aberto ou pseudo-termo).
\end{addmargin}

\vskip 0.5cm


\noindent São exemplos de termos:

\begin{itemize}[itemsep=0.01pt]
\renewcommand\labelitemi{\textbf{-}}
    \item $x$
    \item recursividade
    \item PROLOG
    \item  O autor de Mar Morto
    \item  A linguagem de programação baseada na linguagem ALGOL
    \item $A \cap B$
    \item Maria
    \item 3
    \item (1, 2)
    \item $5 \times 7$
    \item $x + y$
    \item $y!$

\end{itemize}

\vskip 0.5cm

\begin{addmargin}[2.0cm]{0cm}
\textbf{Enunciado} (ou proposição) é a expressão que correlaciona objetos ou descreve propriedades de objetos.
\end{addmargin}

\vskip 0.5cm

\noindent São exemplos de termos:

\begin{itemize}[itemsep=0.01pt]
\renewcommand\labelitemi{\textbf{-}}
    \item PASCAL é uma linguagem de programação de alto nível
    \item $x + y = 3$
    \item  = (3, + (x,y))
    \item  maior(3, 2)
    \item professor(Ivo, Matemática)
    \item ensina(Ivo, João)
    \item Piquet é campeão do mundo
    \item Todo nº par é divisível por 2
    \item ou Pedro estuda ou Pedro será reprovado no nivelamento
    \item $x^2 - 5x + 6 = 0$
    \item pertence(México, América)
\end{itemize}

% Descobrir como fazer essa parte
\textbf{Comentário:} As variáveis $\underline{x}$ e $\underline{y}$, que ocorrem nas expressões $\bm{x + y = 3}$ e $\bm{x^2 - 5x + 6 = 0}$, são variáveis livres em tais expressões.
Tais variáveis atuam como nomes de elementos arbitrários do domínio de discurso.

\noindent
Cabe ter presente que não serão considerados enunciados as expressões sob a forma exclamativa, interrogativa ou imperativa.
Em nossa acepção, um enunciado corresponde às expressões declarativas da linguagem natural.
Um enunciado contendo variáveis livres é chamado de \textbf{enunciado aberto}.
Caso contrário, será chamado de \textbf{enunciado fechado}.

\bigskip

\noindent
Exemplos de enunciados abertos:

\hskip 2.5cm soma(x, y) = 3

\hskip 2.5cm maior(x, 7)

\hskip 2.5cm professor(x, 2º grau)

\bigskip

\noindent
Exemplos de enunciados fechados:

\hskip 2.5cm professora(Maria, 2º grau)

\hskip 2.5cm maior(3, 2)

\hskip 2.5cm $\forall x(corredor(x) \rightarrow resistente(x))$

Em resumo, as expressões bem formadas da linguagem simbólica assim se classificam:

% Melhorar
\[
    \text{Expressões:}
    \begin{cases}
        Termos
        \begin{cases}
            \text{Termo fechado ou nome}\\
            \text{Termo aberto ou pseudo-nome}\\
        \end{cases}

        \\
        \\

        Enunciados
        \begin{cases}
            \text{Enunciado fechado ou sentença}\\
            \text{Enunciado aberto}\\
        \end{cases}
    \end{cases}
\]


\newpage

\section{Símbolos lógicos}

Se for considerado com atenção o vocabulário lógico, é fácil traçar uma distinção superficial entre as verdades lógicas e os enunciados verdadeiros de outras espécies (verdade factual).
Um \textbf{enunciado logicamente verdadeiro} tem esta peculiaridade: partículas básicas como \textbf{não, e, ou, a menos que, se ... então, nem, algum, todo}, etc. ocorrem no enunciado de tal forma que o enunciado é verdadeiro independentemente de seus outros ingredientes.

Um enunciado logicamente verdadeiro é aquele cuja verdade depende exclusivamente do arranjo de certas expressões, ditas vocábulos lógicos, e não de um teste empírico ou observacional.
Esses vocábulos lógicos são: \textbf{não, e, ou, a menos que, se ... então, nem, algum, todo}, etc.
Consideremos os clássicos exemplos:

\bigskip
\noindent
\textbf{Ex. 1.} Sócrates é mortal ou Sócrates não é mortal.

\bigskip
\noindent
A substituição (gramaticalmente adequada) de \textbf{Sócrates} e \textbf{mortal} no exemplo acima é incapaz de tornar esse enunciado falso.
Assim, \textbf{Platão é grego ou Platão não é grego} é igualmente verdadeiro.

\bigskip
\noindent
\textbf{Ex. 2.} Se todo homem é mortal e Sócrates é homem, então Sócrates é mortal.

\bigskip
\noindent
Não somente esta sentença é verdadeira, como também ela é verdadeira independentemente dos constituintes \textbf{homem}, \textbf{mortal} e \textbf{Sócrates}; nenhuma alteração dessas palavras é capaz de transformar a sentença numa falsidade.
\newpage

\noindent Qualquer enunciado da forma:

Se todo \textbf{A} é um \textbf{B} e \textbf{C} é um \textbf{A} então \textbf{C} é um \textbf{B}

\noindent é igualmente verdadeiro, desde que as variáveis sejam corretamente substituídas.

Uma palavra é dita \textbf{ocorrer essencialmente} num enunciado verdadeiro (resp. falso) se a troca dessa palavra por alguma outra for capaz de tornar esse enunciado falso (resp. verdadeiro).
Quando este não for o caso, diz-se que a palavra \textbf{ocorre vacuamente}.
Assim, as palavras 'Sócrates' e 'homem' ocorrem essencialmente no enunciado \textbf{Sócrates é um homem}, visto que o enunciado \textbf{Bucéfalo é um homem} ou \textbf{Sócrates é um cavalo} são falsas.
Por outro lado, \textbf{Sócrates} e \textbf{mortal} ocorrem vacuamente no Ex. 1, e \textbf{Sócrates}, \textbf{homem} e \textbf{mortal} ocorrem vacuamente no Ex. 2.
Mediante esta distinção, podemos agora definir o que entendemos por uma verdade lógica ou enunciado logicamente verdadeiro.

Uma \textbf{verdade lógica} é aquela verdade em que apenas os vocábulos lógicos ocorrem essencialmente.

\noindent \textbf{Ex. 3.} João foi aprovado ou João não foi aprovado

O vocabulário da lógica é básico a todo discurso.
Assim, se listarmos um número suficiente de enunciados (simples ou complexos) da Geologia, por exemplo, veremos que o vocabulário da Lógica aí se encontra.
O mesmo pode ser dito de qualquer disciplina.
As verdades da Lógica são verdades (triviais) da Geologia, Economia, etc.
Isto justifica afirmação de que a Lógica tem uma abrangência universal, sendo o denominador comum das diversas ciências especiais.

As partículas lógicas \textbf{e, ou, não, se ... então, todo} e \textbf{existe} desempenham importante papel no estabelecimento das disciplinas em geral, visto que a partir destas e dos enunciados simples podem ser formados \textbf{enunciados compostos}.

As partículas lógicas são simbolizadas de diversas formas:

\bigskip
\begin{center}
    \large
    \begin{tabular}{l l l l}
        não          & $\sim$   & $\neg$  & \textasciitilde \\ \hline
        e            & $\wedge$ & \& & $\sqcap$ \\ \hline
        ou           & $\vee$   & $\bigvee$  & $\sqcup$ \\ \hline
        se ... então & $\rightarrow$ & $\implies$ & $\sqsubset$  \\ \hline
        todo         & $\forall$  & \small( \small) & \LARGE{$\sqcap$} \\ \hline
        existe       & $\exists$ & E & \LARGE{$\sqcup$}  \\ \hline
    \end{tabular}
\end{center}

\bigskip \noindent
São exemplos de enunciados simples:

\hskip 2.6cm $x > 2$,

\hskip 2.5cm $5 \in \mathbb{N}$,

\hskip 2.5cm $ 3 = x + y$,

\hskip 2.5cm André é programador

\bigskip \noindent
São exemplos de enunciados compostos:

\hskip 2.5cm $\forall x(x > 9 \to x > 3)$,


\hskip 2.5cm $2 \in \mathbb{N} \wedge x = y$,

\hskip 2.5cm $\sim(5 > 2 \vee 7 < 8)$

\newpage

\subsection*{Linguagem natural x Linguagem simbólica}

\end{document}
